% Proof of nose hoover thermostat (slide 140 CP II, audio 9b)





\documentclass[%
paper=a4,
english,
fontsize=11.9pt,
twoside=true,
pointlessnumbers,
BCOR5mm,
titlepage=true,
% smallheadings,
headinclude,
cleardoublepage=empty,
openright,
appendixprefix,
chapterprefix=false,
DIV12
]{scrbook}

\usepackage{wrapfig}
\usepackage[nottoc,numbib]{tocbibind}
\usepackage[T1]{fontenc}
\usepackage[utf8]{inputenc}
%\usepackage[german]{babel}
\usepackage{amssymb}
\usepackage{pifont}
\usepackage{textcomp}
\usepackage[centertags]{amsmath}
\usepackage{graphicx}\graphicspath{{figures/}{logos/}}
\usepackage{booktabs}
\usepackage{url}
\usepackage{epsfig}
\usepackage{textcomp}
\usepackage[font=small,labelfont=bf]{caption}
\usepackage[font=small,labelfont=bf]{subcaption}
\usepackage{lmodern}
\usepackage[squaren]{SIunits}
\usepackage{placeins}
\usepackage{color}
\usepackage{listings}
\usepackage{floatpag}
\usepackage{needspace}
%\usepackage{float}
\usepackage[round]{natbib}
\usepackage{scrpage2}
\floatpagestyle{empty} 
\usepackage{verbatim}
\usepackage{array}
\usepackage{hyperref}

% ======================= Layout Anpassungen =======================

% inhaltsverzeichnis bis 2. ebene
\setcounter{secnumdepth}{2}
\setcounter{tocdepth}{3}

% einfaceh kopfzeile mit kapitel/section mitte; seitenzahl außen
\pagestyle{scrheadings}
\automark[section]{chapter}
\renewcommand*{\chaptermarkformat}{Chapter \thechapter: \autodot\enskip}
\renewcommand*{\sectionmarkformat}{\autodot\enskip}
\ohead{\pagemark}
\chead{\headmark}
\cfoot{}
\ofoot{\pagemark}
\ifoot{}
\setheadsepline{0.4pt}

%\renewcommand{\topfraction}{.85}
%\renewcommand{\bottomfraction}{.7}
%\renewcommand{\textfraction}{.10}
%\renewcommand{\floatpagefraction}{.8}

\renewcommand{\topfraction}{.30}
\renewcommand{\bottomfraction}{.22}
\renewcommand{\textfraction}{.12}
\renewcommand{\floatpagefraction}{.12}





% figure nummeririerung: kapitel.figure
\renewcommand{\thefigure}{\thechapter.\arabic{figure}}

% bildunterschriften etwas kleiner, keine einrückung
\setkomafont{captionlabel}{\normalfont\small}
\setkomafont{caption}{\normalfont\small}
\setcapindent{0pt}

% kein ausgleich des unteren randes bei twoside
\raggedbottom

% schusterjungen und hurenkinder verhndern
\clubpenalty = 10000
\widowpenalty = 100000
\displaywidowpenalty = 10000

% listings setup fuer C/C++ IDL




\definecolor{mygreen}{rgb}{0,0.6,0}
\definecolor{mygray}{rgb}{0.5,0.5,0.5}
\definecolor{mymauve}{rgb}{0.58,0,0.82}

\lstset{ %
  backgroundcolor=\color{white},   % choose the background color; you must add \usepackage{color} or \usepackage{xcolor}
  basicstyle=\footnotesize,        % the size of the fonts that are used for the code
  breakatwhitespace=false,         % sets if automatic breaks should only happen at whitespace
  breaklines=true,                 % sets automatic line breaking
  captionpos=b,                    % sets the caption-position to bottom
  commentstyle=\color{mygreen},    % comment style
  deletekeywords={...},            % if you want to delete keywords from the given language
  escapeinside={\%*}{*)},          % if you want to add LaTeX within your code
  extendedchars=true,              % lets you use non-ASCII characters; for 8-bits encodings only, does not work with UTF-8
  frame=single,                    % adds a frame around the code
  keepspaces=true,                 % keeps spaces in text, useful for keeping indentation of code (possibly needs columns=flexible)
  keywordstyle=\color{red},       % keyword style
  language=C,                 % the language of the code
  morekeywords={*,...},            % if you want to add more keywords to the set
  numbers=left,                    % where to put the line-numbers; possible values are (none, left, right)
  numbersep=5pt,                   % how far the line-numbers are from the code
  numberstyle=\tiny\color{mygray}, % the style that is used for the line-numbers
  rulecolor=\color{black},         % if not set, the frame-color may be changed on line-breaks within not-black text (e.g. comments (green here))
  showspaces=false,                % show spaces everywhere adding particular underscores; it overrides 'showstringspaces'
  showstringspaces=false,          % underline spaces within strings only
  showtabs=false,                  % show tabs within strings adding particular underscores
  stepnumber=1,                    % the step between two line-numbers. If it's 1, each line will be numbered
  stringstyle=\color{mymauve},     % string literal style
  tabsize=2,                       % sets default tabsize to 2 spaces
  title=\lstname                   % show the filename of files included with \lstinputlisting; also try caption instead of title
}



%\lstset{frameround=tttt}
%\lstset{frameround=trbl}

%%% ==============================================================


% eigene makrodefinitionen
\input{makros}

%\newcommand{\fixme}[1]{\textcolor{red}{(#1)}%
%  \marginpar{\textcolor{red}{Fix Me!}}%
%}
%\newcommand{\todo}[1]{\textcolor{blue}{(#1)}%
%  \marginpar{\textcolor{blue}{Todo}}%
%}

        

%%%%%%%%%%%%%%%%%%%%%%%%%%%%%%%%%%%%%%%%%%%%%%%%%%%%%%%%%%%%%%%%%%%%%%%%%%%%%%
\begin{document}


\title{Computational Statistical Physics}
\subtitle{\emph{Lecture Notes}}
\author{Nicola Offeddu \\Marcel Thielmann \\Madis Ollikainen}
\date{Summer Semester 2014}

\frontmatter
\begin{titlepage}
  \include{chapters/shit/titelseite}
\end{titlepage}

\cleardoublepage
\include{chapters/shit/preface}
\cleardoublepage

\renewcommand{\contentsname}{Table of contents}
\tableofcontents
\cleardoublepage

\mainmatter


%%%%%%%%%%%%%%%%%%%%%%%%%%%%%%%%%%%%%%%%%%%%%%%
% Chapters
%%%%%%%%%%%%%%%%%%%%%%%%%%%%%%%%%%%%%%%%%%%%%%%
%%%%%%%%%%%%%%%%%%%%%%%%%%%%%%%%%%%%%%%%%%%%%%%
\include{chapters/monte_carlo/classical_stat_mech}


\section{Monte Carlo Algorithms}

Monte Carlo integration has been already extensively discussed in ICP (see \citep{comp_phys}). In this section, we will briefly summarize how this method works and go deeper into some details and properties we did not study in the past semester. 

\vspace{0.2cm}
\noindent
The basic idea behind Monte Carlo methods is that in order to calculate some thermodynamical quantity, it is enough to sample randomly in phase space instead of averaging over all states. If the sampling is large enough, the computed quantity will eventually converge toward the real thermodynamical quantity. The main steps in the Monte Carlo integration are:

\begin{itemize}
\item Choose randomly a new configuration in phase space (with a Markov chain).
\item Accept or reject the new configuration, depending on the strategy used (e.g., Glauber Dynamics).
\item Calculate the physical quantity and add it in the averaging loop.
\item Repeat the procedure.
\end{itemize}


\subsection{Markov Chains in Monte Carlo: M(RT)$^2$, Glauber, Kawasaki and Creutz algorithms}


\vspace{0.1cm}
\noindent
\begin{minipage}{\textwidth}
\begin{minipage}{.48\textwidth}
Often, choosing equally distributed configurations will be very inefficient since most of the possible configurations are unlikely to be assumed by the system. As an example take the kinetic energy of a gas. The distribution of the average energy will be a sharp peak, as in Fig. \ref{fig:sampling}. There are a number of methods to avoid unnecessary sampling of regions where the system is unlikely to be found (see \emph{importance sampling} in \citet{comp_phys} as an example).
\end{minipage}%
\hfill
\begin{minipage}{.48\textwidth}
  \centering
  \includegraphics[height=150pt]{pics/sampling}
  \captionof{figure}{Example of energy distribution}
  \label{fig:sampling}
\end{minipage}
\end{minipage}
\vspace{0.1cm}


\begin{comment}


\begin{wrapfigure}{r}{0.5\textwidth}
  	\begin{center}
    	\includegraphics[width=0.5\textwidth]{pics/sampling}
		\label{fig:sampling}
  	\end{center}
  	\caption{Example of energy distribution}
\end{wrapfigure}

Often, choosing equally distributed configurations will be very inefficient since most of the possible configurations are unlikely to be assumed by the system. As an example take the kinetic energy of a gas. The distribution of the average energy will be a sharp peak, as in Fig. \ref{fig:sampling}. There are a number of methods to avoid unnecessary sampling of regions where the system is unlikely to be found (see \emph{importance sampling} in \citet{comp_phys} as an example). 

\end{comment}


A common way to choose the configurations to sample is introducing a virtual time $\tau$ and explore the phase space through a \emph{Markov chain}. Mind that the time $\tau$ has no physical meaning, and it only represents the steps of a stochastic process. 

In terms of a Markov chain, the transition probability from one state to another is given by the probability of a new state to be proposed ($T$) and the probability of this state to be accepted and assumed ($A$). In other words, $T(X\rightarrow Y)$ simply gives us the probability that a new configuration $Y$ is proposed, starting from the configuration $X$. This must fulfill three conditions:
\begin{enumerate}
\item \emph{Ergodicity}: any configuration in the phase space must be reachable within a finite number of steps
\item \emph{Normalization}: $\sum_Y{T(X\rightarrow Y)} =1$
\item \emph{Reversibility}:  $T(X\rightarrow Y)= T(Y\rightarrow X)$
\end{enumerate}
Once a configuration is proposed, one can accept the new configuration with probability $A(X\rightarrow Y)$ or reject it with probability $1- A(X\rightarrow Y)$. The transition probability is then given by
\begin{equation}
W(X\rightarrow Y) = T(X\rightarrow Y) \cdot A(X\rightarrow Y). 
\end{equation}
With the transition probability one can investigate the probability of finding the state in a certain configuration (during the stochastic process, not in real time!) $p\kl{X,\tau}$. The \emph{master equation} describes how the distribution evolves in time. 
\begin{equation}
\der{p\kl{X,\tau}}{\tau}=\sum_Y{p(Y)W(Y\rightarrow X)}  -\sum_Y{p(X)W(X\rightarrow Y) }
\label{eq:master}
\end{equation}
For Markov chains, it is known that the system always reaches a stationary state (called $p_{\text{st}}$) defined by the derivative in Eq. \eqref{eq:master} being zero. The transition probability must fulfill
\begin{enumerate}
\item \emph{Ergodicity}: any configuration must be reachable: $\forall X,Y:$ $W(X\rightarrow Y)\ge0$
\item \emph{Normalization}: $\sum_Y{W(X\rightarrow Y)} =1$
\item \emph{Homogeneity}:  $\sum_Yp_{\text{st}}(Y)W(Y\rightarrow X)= p_{\text{st}}(X)$
\end{enumerate} 
 
Note that the condition of reversibility is not required anymore. This is one of the effects of introducing $A\kl{X\rightarrow Y}$. Just think of a two level system, in which one of the two energy levels is higher (e.g. the electronic shells in an atom): At low energies it would be nonsense to equally sample the excited and the ground state of the electrons. On the contrary, at very high energies the sampling will have to reflect the higher probability of being in an excited state, rather then in the ground state. In order for the Markov chain algorithm to choose effectively which areas of the phase space to explore, somehow $W$ has to depend on the system. Imposing the distribution of the stationary states $p_\text{st}$ as the equilibrium distribution of the physical system $p_\text{st}$ (a real and measurable distribution) is called \emph{detailed balance}:
\begin{equation}
\der{p\kl{X,\tau}}{\tau}=0 \Leftrightarrow p_{\text{st}} \overset{!}{=}  p_{\text{eq}}
\label{eq:detailed_balance}
\end{equation}
It then follows from the stationary state condition ($\der{p\kl{X,\tau}}{\tau}=0$) that
$$\sum_Y{p_{\text{eq}}(Y)W(Y\rightarrow X)}  = \sum_Y{p_{\text{eq}}(X)W(X\rightarrow Y)}.$$ 
A sufficient (not necessary!) condition for this to be true is
 \begin{equation}
 {p_{\text{eq}}(Y)W(Y\rightarrow X)}  = {p_{\text{eq}}(X)W(X\rightarrow Y)}.
 \label{eq:detailed_balance2}
 \end{equation}
As an example, in a canonical ensemble (at fixed Temperature T), the equilibrium distribution is given by the Boltzmann factor: $p_{	\text{eq}}(X)= \frac{1}{Z_T}\text{exp}\ekl{-\frac{E(X)}{k_BT}}$ with the partition function $Z_T=\sum_X{ \text{exp}\ekl{-\frac{E(X)}{k_BT}} }$. The Boltzmann equilibrium indeed fulfills the detailed balance (see M(RT)$^2$ algorithm).




\subsubsection*{M(RT)$^2$ algorithm:}

If equation \eqref{eq:detailed_balance2} is fulfilled, we automatically found the way to fulfill detailed balance by the chosen $W$ and $p_{	\text{eq}}$. The algorithm (also called Metropolis algorithm\footnote{The rather curious name of this algorithm finds its reason in the names of the authors of the paper in which it was proposed: \citet{mrtrt}. \emph{RT} is squared because besides Metropolis, the other four authors of the paper formed two married couples and therefore carried the same family names. The real contributions of some of the authors (in particular of Metropolis and of A.H. Teller) is subject of controversy \citep{controversymrtrt,controversymrtrt2}. It has been even stated by Roy Glauber and Emilio Segr\'e that the original algorithm was invented by Enrico Fermi, which described it to Metropolis while they were working together at Los Alamos and later reinvented by Stan Ulam \citep{segre}.}) uses the acceptance probability 
\begin{equation}
A\kl{X\rightarrow Y} = min\ekl{1,\frac{p_{	\text{eq}}\kl{Y}}{p_{	\text{eq}}\kl{X}}}.
\label{eq:metropolis}
\end{equation}
In the case of the canonical ensemble with $p_{	\text{eq}}\kl{X} = \frac{1}{Z_T}\text{exp}\ekl{-\frac{E(X)}{k_BT}} $ the acceptance probability becomes then
\begin{equation}
A\kl{X\rightarrow Y} = min\ekl{1,\text{exp}\ekl{-\frac{E(Y)-E(X)}{k_BT}} }=min\ekl{1,\text{exp}\ekl{-\frac{\Delta E}{k_BT}} },
\end{equation}
which means that if the energy decreases, the step is always accepted, and if the energy increases it is accepted with probability $\text{exp}\ekl{-\frac{\Delta E}{k_BT}}$. Plugging Eq. \eqref{eq:metropolis} with $p_{	\text{eq}}$ into Eq. \eqref{eq:detailed_balance2} shows that detailed balance is fulfilled. For a more detailed discussion about the Metropolis and alternatives algorithms (e.g. Glauber dynamics), see \citet{comp_phys}. The algorithm has been then generalized in 1970 \citep{mrtrtgeneral}.
We can use this rather general algorithm to explore the phase space of the Ising model, by flipping the values on the lattice following the acceptance probability. Summarized, the steps in the Metropolis algorithm would then be:

\begin{itemize}
\item Choose (randomly) a site $i$
\item Calculate $\Delta E=E(Y)-E(X)=2J\sigma_i h_i$
\item If $\Delta E\leq0$, flip the spin, otherwise accept it with probability $\text{exp}\ekl{-\frac{\Delta E}{k_BT}}$
\end{itemize}
with $h_i=\sum_{i,j:nn}{\sigma_j}$ and $E=-J\sum_{i,j:nn}{\sigma_i\sigma_j}$. Mind that in the case of a squared lattice there are a limited number of possibilities that can occur. Consider creating a look-up table to spare computation time during the acceptance loop. For a 2D lattice, $h_i \in \mkl{0, \pm 2, \pm 4}$.
 
 
 \subsubsection*{Heat bath method (Glauber dynamics):}
The Metropolis algorithm is not the only possible Monte Carlo update: there are a number of other algorithms that fulfill Eq. \eqref{eq:detailed_balance2}. One of these has been elaborated by Glauber, with the acceptance probability given as

 \begin{equation}
A\kl{X\rightarrow Y} \equiv \frac{ \text{exp}\ekl{ - \frac{\Delta E}{k_B T} }}  {    1 + \text{exp}\ekl{ - \frac{\Delta E}{k_B T}}     }
\label{eq:glauber}
\end{equation}

One can see that the expression in Eq. \eqref{eq:glauber} fulfills Eq. \eqref{eq:detailed_balance2}:
 
 
 \begin{align*}
1 &= 1 \\
\Leftrightarrow  \frac{ 1 + \text{exp}\ekl{ - \frac{\Delta E}{k_B T} }  }  {    1 + \text{exp}\ekl{ - \frac{\Delta E}{k_B T}} } &= \text{exp}\ekl{ -\frac{\Delta E}{k_B T} }   \text{exp}\ekl{ +\frac{\Delta E}{k_B T} }  \\
\Leftrightarrow    \frac{ \kl{1 + \text{exp}\ekl{ +\frac{\Delta E}{k_B T}} }\text{exp}\ekl{ - \frac{\Delta E}{k_B T} }}  {    1 + \text{exp}\ekl{ - \frac{\Delta E}{k_B T}}     }  &= \text{exp}\ekl{\frac{ E_X-E_Y}{k_B T} }   \text{exp}\ekl{ +\frac{\Delta E}{k_B T} }   \\
\Leftrightarrow    \kl{1 + \text{exp}\ekl{ +\frac{\Delta E}{k_B T}} }\frac{ \text{exp}\ekl{ - \frac{\Delta E}{k_B T} }}  {    1 + \text{exp}\ekl{ - \frac{\Delta E}{k_B T}}     }  &= \frac{\text{exp}\ekl{\frac{ E_X}{k_B T} }}{\text{exp}\ekl{ \frac{E_Y}{k_B T} } }     \text{exp}\ekl{ +\frac{\Delta E}{k_B T} }   \\
\Leftrightarrow  \text{exp}\ekl{ \frac{E_Y}{k_B T} }    \frac{ \text{exp}\ekl{ - \frac{\Delta E}{k_B T} }}  {    1 + \text{exp}\ekl{ - \frac{\Delta E}{k_B T}}     }  &= \text{exp}\ekl{\frac{ E_X}{k_B T} }    \frac{ \text{exp}\ekl{ +\frac{\Delta E}{k_B T} }}  {    1 + \text{exp}\ekl{ +\frac{\Delta E}{k_B T}}     } \\
\underset{\text{const. temperature}}{\overset{T(X\rightarrow Y)\text{ symmetric}}{\Leftrightarrow}}  {p_{\text{eq}}(Y)T(Y\rightarrow X) A_{Gl.}(Y\rightarrow X)}  &= {p_{	\text{eq}}(X)T(X\rightarrow Y) A_{Gl.}(X\rightarrow Y)}.
\end{align*}
 
 


 
\noindent 
Mind that knowledge about the whole system configuration before the spin flip is not needed here: only the local configuration around the site is relevant. With $J = 1$, the probability to flip the spin $\sigma_i$ is:
$$
A_i = \frac{\text{exp}\ekl{\frac{-2 \sigma_i h_i}{k_BT}}}{1+\text{exp}\ekl{\frac{-2 \sigma_i h_i}{k_BT}}}
$$
with $h_i$ being the local field as usual $h_i = \sum_{j=nn}{\sigma_j}$. Using the abbreviation $p_i \equiv \frac{\text{exp}\ekl{2 \beta h_i}}{1+\text{exp}\ekl{2 \beta h_i}}$, one can express the probability to flip the spin as being

\begin{equation}
p_{\text{flip}} =\begin{cases}
  p_i  & \text{for }\sigma_i=-1\\
  1-p_i & \text{for }\sigma_i=+1
\end{cases}
\text{\hspace{0.5cm} and  \hspace{0.5cm}}\hfill 
p_{\text{no flip}}   =\begin{cases}
  1- p_i  & \text{for }\sigma_i=-1\\
  p_i & \text{for }\sigma_i=+1
\end{cases}
\end{equation}

\noindent
This can be implemented as  $$\sigma_i(\tau+1)=-\sigma_i(\tau)\cdot sign(A_i-z),$$ with $z$ being a random number, or as
$$
\sigma_i(\tau+1) = \begin{cases}
  +1  & \text{with propability }p_i\\
  -1 & \text{with propability }1- p_i
\end{cases}
\text{\hspace{0.5cm} with  \hspace{0.5cm}}\hfill 
p_i \equiv \frac{\text{exp}\ekl{2 \beta h_i}}{1+\text{exp}\ekl{2 \beta h_i}}
$$

This method which does not depend on the spin at time $t$, is called \emph{heat bath MC}.


 
 
\subsubsection*{Binary mixtures (Kawasaki dynamics):}
In this method, the sum of the spins pointing up and the sum of the spins pointing down (i.e., the magnetization) is held constant. Kawasaki dynamics can be used for simulating  binary mixtures of gases and other systems were the population numbers are conserved. In the case of a two species mixture, the energy is larger for A-B bonds, with A and B being the two species in the binary mixture (spin up and spin down particles, two different gas molecules, etc.). What  one can do is to switch two particles with a certain probability and then add the configuration to the averaging loop. 

\subsubsection*{Creutz algorithm:}
Let us consider a situation in which energy is constant. The algorithm generally used in this case is the \emph{Creutz} algorithm. In this technique the condition of energy conservation is relaxed a bit and energy is not exactly conserved anymore. The movement in phase space is therefore not strictly constrained to a subspace of constant energy, but we have a certain additional volume in which we can freely move. The condition is softened by introducing a \emph{demon}, which is a small reservoir of energy $E_d$ that can store a certain maximum energy $E_m$:

\begin{itemize}
\item Choose a site
\item Calculate $\Delta E$ for the spin flip
\item Accept the change if $E_m\ge E_d-\Delta E\ge 0 $
\end{itemize}
This method is completely deterministic (besides the fact that one can randomly choose the sites to update) and therefore reversible. The drawback is that the temperature of the system is not known, but it can be estimated  with the Boltzmann distribution. Taking a histogram of the energies $E_d$ one observes a distribution $P(E_d)\propto \text{exp}\ekl{ -\frac{E_d}{k_BT} }$. The fit is not optimal, since one has very few values of $E_d$. The bigger $E_m$, the faster is the method, since the condition of constant energy is relaxed and the exploration of phase space is less restricted to certain regions. 



\subsubsection*{Q2R:}
In the case of $E_m\rightarrow0$, the Creutz algorithm becomes a totalistic cellular automaton called \emph{Q2R}. The update rules on a square lattice are then given by
 $$
 \sigma_{i,j}(\tau+1) = f(x_{ij})\oplus\sigma_{i,j}(\tau)
 $$
with 
$$
x_{i,j}= \sigma_{i-1,j} +\sigma_{i+1,j} +\sigma_{i,j-1} +\sigma_{i,j+1}
\text{\hspace{0.5cm} and  \hspace{0.5cm}}\hfill 
f(x)=\begin{cases}
  1  & \text{if }x=2\\
  0 & \text{if }x\neq 2
\end{cases}
$$
In this case, the spins are flipped if and only if the change in energy is zero. This can be implemented in a very efficient way using multi spin coding. The changer word defined by $f(x)$ can be expressed in a very elegant way using logical functions:
$$
\sigma(\tau+1) = \sigma(\tau)\oplus\mkl  { \ekl{\kl{\sigma_1\oplus \sigma_2}  \wedge  \kl{\sigma_3\oplus \sigma_4} }  \lor   \ekl{\kl{\sigma_1\oplus \sigma_3}  \wedge  \kl{\sigma_2\oplus \sigma_4} } }
$$
This logical expression can be computed in roughly 12 cycles (8 logical operation and 4 fetches) which last typically around $10$ns. At this computational speed and using multi spin coding (see Sec. \ref{sec:multi_spin}) one can roughly update 4 sites per nanosecond. The method is extremely fast, deterministic and reversible. The problem is that it is not ergodic, and that it strongly depends on the initial configuration. As an example, try to imagine the development of a small lattice, in which only $\sigma_{2,1}$ and $\sigma_{1,2}$ are equal to +1. In fact, this method is not used in statistical physics but it is useful for other purposes e.g., neuroinformatics or cellular automata.




 \subsection{Boundary conditions:}
 
When simulating a lattice, one of the finite size effects that one has to take into account is that at the boundaries the sites do not have further neighbors. The values there can be fixed (boundary condition) or periodic boundaries can be introduced. Mind that this is not only a question of finite size effects, but it can also correspond to some real physical situation. As an example, think of some zero potential boundary condition while solving the electrostatic potential using finite difference methods\footnote{See \citet{comp_phys}}. In finite lattices following methods can be used:

\begin{itemize}
\item Open boundaries, i.e. no neighbors at the edges of the system.
\item Fixed boundary conditions,
\item and periodic boundaries.
\end{itemize}

If our system is big enough\footnote{\emph{Big} here is not purely arbitrary but it depends on what we want to simulate. A good measure of \emph{big} can be that the edges of the system are uncorrelated. It is clear that this method is useless in \emph{small} lattices.} one identies the two sides of the lattice as being neighbors. ($\sigma_{i,L+1}\equiv\sigma_{i,1}$, and $\sigma_{L+1,j}\equiv\sigma_{1,j}$). Identifying the last element of a row (or column) with the first element of the next row (or column) leads to so called \emph{helical boundary condition}. $\sigma_{i,L+1}\equiv\sigma_{i+1,1}$, therefore one can use only one index $k = i+j(L-1)$ instead of keeping track of two.



\subsection{Sampling Uncorrelated Configurations}
Each time we accept a spin-flip in our sampling chain, a new configuration is generated. The problem is that the new and the previous configurations are strongly correlated, and the error analysis (e.g., the decreasing of the error like $\propto \frac{1}{\sqrt{N}}$) is no longer valid. We have to find a measure to know whether we already are in equilibrium or not and to be sure that we are sampling uncorrelated configurations. The (Monte Carlo) time evolution of a quantity is defined as

\begin{equation}
\avkl{A(\tau)}= \sum_X{p\kl{X,\tau}A\kl{X\kl{\tau}}} \overset{\text{eq.}}{=} \sum_X{p\kl{X,\tau_0}A\kl{X(\tau)}}
\end{equation} 




\vspace{0.1cm}
\noindent
\begin{minipage}{\textwidth}
\begin{minipage}{.48\textwidth}

With the condition expressed in \eqref{eq:master}, we know how the probability distribution $p$ evolves. If we assume that our configuration distribution is not at equilibrium at $\tau = \tau_0$, we can define the \emph{non-linear correlation function}:
\begin{equation}
\Phi_A^{nl}\kl{\tau} = \frac {\avkl{A(\tau)}- \avkl{A(\infty)}}  {\avkl{A(\tau_0)}- \avkl{A(\infty)}}
\end{equation}
This is not a correlation function in the strict sense, but it can be a measure to investigate the correlation of the configurations. 
\end{minipage}%
\hfill
\begin{minipage}{.48\textwidth}
  \centering
  \includegraphics[height=150pt]{pics/non_lin_corr_fun}
  \captionof{figure}{Non-linear correlation function over Monte Carlo time.}
  \label{fig:non_lin_corr_fun.pdf}
\end{minipage}
\end{minipage}
\vspace{0.1cm}



The \emph{non-linear} correlation time $\tau_A^{nl}$ describes the relaxation towards equilibrium:
\begin{equation}
\tau_A^{nl} \equiv \int_0^{\infty} \Phi_A^{nl}\kl{\tau} d\tau
\end{equation}
Near $T_c$, it assumes the form of a power law (\emph{critical slowing down}):
\begin{equation}
\tau_A^{nl} \propto \abs{T-T_c}^{-z_A^{nl}}
\end{equation} with $z_A^{nl}$ being the non-linear dynamical critical exponent. This means that at $T_c$, the time needed to reach equilibrium diverges!






The linear correlation function of two values $A$, $B$
\begin{equation}
\Phi_{AB}\kl{\tau} = \frac {\avkl{A(\tau_0)B(\tau)}- \avkl{A}\avkl{B}}  {\avkl{A B}- \avkl{A}\avkl{B}}
\label{eq:lin_corr_fun}
\end{equation} with $$ \avkl{A\kl{\tau_0} B\kl{\tau}}     =  \sum_X{   p\kl{X,\tau_0} A\kl{X\kl{\tau_0}} B \kl{X\kl{\tau}}     }$$
is now a proper correlation function. Note that it goes from 1 to 0 when $\tau$ goes to infinity. If $A=B$, we call Eq. \eqref{eq:lin_corr_fun} the \emph{autocorrelation function}. For the spin-spin correlation in the Ising model we get:

$$\Phi_{\sigma}\kl{\tau} = \frac {\avkl{\sigma(\tau_0)\sigma(\tau)}- \avkl{\sigma(\tau_0)}^2} {\avkl{\sigma^2(\tau_0)}- \avkl{\sigma(\tau_0)}^2}$$

The \emph{linear} correlation time $\tau_A^{nl}$ describes the relaxation in equilibrium:
\begin{equation}
\tau_{AB} \equiv \int_0^{\infty} \Phi_{AB}\kl{\tau} d\tau
\end{equation}

  \begin{center}
  \includegraphics[width=0.7\textwidth]{pics/spin_spin_fun}
  \captionof{figure}{Spin-spin autocorrelation function over MC time in the Ising model.}
  \label{fig:spin_spin_fun}
\end{center}

\begin{comment}

\vspace{0.1cm}
\noindent
\begin{minipage}{\textwidth}
\begin{minipage}{.48\textwidth}
\noindent
If $A=B$, we call Eq. \eqref{eq:lin_corr_fun} the \emph{autocorrelation function}. For the spin-spin correlation in the Ising model we get:

$$\Phi_{\sigma}\kl{\tau} = \frac {\avkl{\sigma(\tau_0)\sigma(\tau)}- \avkl{\sigma(\tau_0)}^2} {\avkl{\sigma^2(\tau_0)}- \avkl{\sigma(\tau_0)}^2}$$

\end{minipage}%
\hfill
\begin{minipage}{.48\textwidth}
  \centering
  \includegraphics[height=150pt]{pics/spin_spin_fun}
  \captionof{figure}{Spin-spin autocorrelation function over MC time in the Ising model.}
  \label{fig:spin_spin_fun}
\end{minipage}
\end{minipage}
\vspace{0.1cm}

\end{comment}



Near $T_c$, it assumes the form of a power law (\emph{critical slowing down}):
\begin{equation}
\tau_{AB} \propto \abs{T-T_c}^{-z_A}
\end{equation} with $z_A$ being the \emph{linear} dynamical critical exponent.




The dynamical exponents turn out to be 
$$z_\sigma = 2.16 \text{ (in 2D)}$$ $$z_\sigma = 2.09 \text{ (in 3D)}$$ There is a conjectured relation between the critical exponents of the previous sections and the critical dynamical exponents (for spin and for energy correlation) that is numerically well established but not yet analytically proven:

\begin{align}
z_\sigma - z_\sigma^{nl} &= \beta \\ 
z_E - z_E^{nl} &=1- \alpha\\
\end{align}


\subsubsection*{Decorrelated configurations:}


\vspace{0.1cm}
\noindent
\begin{minipage}{\textwidth}
\begin{minipage}{.48\textwidth}
\noindent

The behavior we studied until now is only valid in the case of an infinite lattice. In a finite system there is no divergence by definition (See \citet{comp_phys}). The correlation length diverges at the critical temperature $T_c$:
\begin{equation}
L=\xi\kl{T}\propto \abs{T-T_c}^{-\nu}
\label{eq:corr_len_div}
\end{equation}
$$\Rightarrow \tau_{AB} \propto \abs{T-T_c}^{-z_{AB}}\propto L^{\frac{z_{AB}}{\nu}} $$ 
which means that with growing system size, the number of samples to be discarded also increases.


\end{minipage}%
\hfill
\begin{minipage}{.48\textwidth}
  \centering
  \includegraphics[height=150pt]{pics/finite_size}
  \captionof{figure}{Finite size effects for different system sizes can be used to obtain the critical temperature by extrapolation. See \citet{comp_phys}}
  \label{fig:finite_size}
\end{minipage}
\end{minipage}
\vspace{0.1cm}

\noindent
This is a problem when sampling big systems since the computation becomes very expensive. To be sure not to sample correlated configurations one should 
\begin{itemize}
\item reach equilibrium first  (discard $n_0 = c \tau^{nl}(T)$ configurations).
\item Sample every $n_e^{th}=c \tau(T)$ configurations.
\item At $T_c$ use $n_0 = c L^{\frac{z^{nl}}{\nu}}$ and $n_e=c L^{\frac{z}{\nu}}$
\end{itemize}
where $c \approx 3$ is a "safety factor" to make be sure to discard enough samples. A trick for reducing this effect is using cluster algorithms, which we will encounter later on.


\include{chapters/monte_carlo/finite_size_methods}
\include{chapters/monte_carlo/cluster_algorithms}
\include{chapters/monte_carlo/histo_methods}


\section{Renormalization Group}

In this section, we won't be able to go very deeply into the theory of the subject due to its strong mathematical nature. We will therefore only skim through the main concepts and present some numerical results, for more details see \citet{renorm_leeuwen}.

We will make use of symmetries of the physical system to improve our simulations. Usually, the more information there is available about a system, the better certain quantities can be evaluated. Close to the criticality, changes in the scale of the system can be used to better extrapolate the values to an infinite system. Close to the critical point, one of the most important properties is the invariance under conformal transformations. This also implies the invariance under changes of scale (dilation invariance). Mind that this is only true around $T_c$. To increase mathematical precision one has to define exactly what it means to \emph{change scale}. In the case of geometrical objects, the meaning is somewhat intuitive. A good example is the density in fractals, which does not change under scale transformations (remember the sand-box method in \citet{comp_phys}). However, we wikk also treat systems with different energy and spin configurations as well as different Hamiltonians. One possibility is to look at the free energy density, and its invariance. To renormalize a system means to change its scale by a factor $l$: $\tilde{L}= L/l$.  This can be done either in position, or in momentum (Fourier) space.

\subsection{Real Space Renormalization}




If the system is invariant under a certain transformation, in theory one is able to iterate this transformation infinitely often without changing the quantities in the system\footnote{[...]\emph{We can construct certain transformations on the Hamiltonian of the system for which the critical point is a fixed point. Meaning that observables do not change, no matter how many times we apply these transformations.} \citep{wilson_renorm}}. In other words, the critical point is a \emph{fixed point} for this transformation. In order to put the concept of renormalization in a mathematical framework we will give two concrete examples (renormalization of the free energy and decimation of the 1D Ising model). After that, in sec. \ref{subsec:generalization}, we will generalize the concept and present the implementation of renormalization with MC in sec \ref{subsec:MCRG}. 


\subsubsection*{Renormalization and free energy}
We consider the free energy density of a system, and require it to stays invariant under scale transformations. Since the free energy is an \emph{extensive} quantity\footnote{\emph{Extensive} quantities change with the size of the system, like the volume or the total mass of a gas. \emph{Intensive} quantities are not dependent on the system size, e.g., the energy density or the temperature.}, we have to renormalize the quantity itself to the system size. With $\tilde F$ being the renormalized free energy,
\begin{equation}
\tilde{F}\kl{\tilde{\epsilon},\tilde{H}}=l^{-d} F\kl{\epsilon, H}\text{ with }\epsilon\equiv T-T_c.
\end{equation}
We then make use of the scaling law close to the critical point:
$$
F\kl{\epsilon,H} = l^d F\kl{l^{y_T}\epsilon, l^{y_H}H}
$$
$$
\Rightarrow 
\tilde{F}\kl{\tilde{\epsilon},\tilde{H}} = l^d F\kl{l^{y_T}\epsilon, l^{y_H}H}
$$
\begin{equation}
\Rightarrow 
\tilde{\epsilon} = l^{y_T}\epsilon\text{, }\tilde{H} = l^{y_H}H.
\end{equation}

Since renormalization also affects the correlation length $\xi\propto \abs{T-T_c}^{-\nu}= \abs{\epsilon}^{-\nu}$ we can relate the critical exponent $\nu$ to $y_T$:
$$
\tilde{\epsilon}^{-\nu}\propto \tilde{\xi} = \frac{\xi}{l}
$$
$$
\Rightarrow
\tilde{\epsilon} =\frac{\epsilon}{l^{-\frac{1}{\nu}}} =l^{\frac{1}{\nu}}\epsilon=l^{y_T}\epsilon
$$
$$
\Rightarrow
y_T=\frac{1}{\nu}.
$$
The critical point is a fixed point of the transformation: at $T_c$, $\epsilon=0$, and whatever the value of the scaling factor, $\epsilon$ does not change.

\subsubsection*{Majority rule}


An easy example that can be regarded as renormalization of a spin systems is the \emph{majority rule}. One considers local groups of spins $\sigma_i$, and instead of considering them all separately taking a mean over that local group of spins: $\tilde{\sigma}_{\tilde{i}} =\text{sign}\kl{\sum_{\text{region}}\sigma_i}$ (see Fig. \ref{fig:majority_rule}). Attention must be paid on how the transformation is done. In a one dimensional lattice with up/down spins for example it would be an error to apply this transformation on an even number of spins, since the sign is then not well defined for all the possible situations. The fact that one deals with system of a finite size is also something that has to be taken into account: one can renormalize up to a certain scale, before finite size effects are visible.



\noindent
\begin{minipage}{\textwidth}
  \centering
  \includegraphics[width=0.5\textwidth]{pics/majority_rule}
  \captionof{figure}{Majority rule for a 2D Ising model. The spins are grouped into larger regions and averaged. If the majority of the spins is in a certain state, that region is considered as a single spin in that state.}
  \label{fig:majority_rule}
\end{minipage}




\subsubsection*{Decimation of 1D Ising Model}


Another possible rule is \emph{decimation} (see Fig. \ref{fig:decimation}). In decimation one just eliminates certain spins, generally in a regular pattern. 
\begin{figure}[h!]
  \centering
  \includegraphics[height=90pt]{pics/decimation}
  \captionof{figure}{Decimation of spins: every second spin in every direction is ignored.}
  \label{fig:decimation}
\end{figure}
\vspace{0.2cm}



As already mentioned, we will consider a one dimensional Ising chain as a practical example. The spins that only interact with their nearest neighbors:

\vspace{0.2cm}

\noindent
\begin{minipage}{\textwidth}
  \centering
  \includegraphics[height=35pt]{pics/1d_decimation}
 % \captionof{figure}{}
  \label{fig:1d_decimation}
\end{minipage}

\vspace{0.2cm}

To see explicitly what happens with the system, we calculate the partition function $Z$. We will split the partition function into two terms: one for the even sites and one for the odd sites. With $K=-\frac{J}{K_BT}$  and using

\begin{align*}
A &= \sum_{s_{2n+1}=\pm 1}{  e^  {K\kl   {  s_{2n}s_{2n+1}  + s_{2n+1} s_{2n+2}     }  } }\\
&=e^  {K\kl   {  s_{2n}+s_{2n+2}    }  }  + e^  {-K\kl   {  s_{2n}+s_{2n+2}    }  } \\
&=\text{cosh}\ekl{K\kl   {  s_{2n}+s_{2n+2}    } }
\end{align*}

we get
\begin{align*}
\mathcal{Z} &= \sum_{s_{2i}=\pm 1}{  \prod_{2i}{   \ekl{ {  
\sum_{s_{2i+1}=\pm 1}{  \prod_{2i+1}{e^  {K\kl   {  s_{2i}s_{2i+1}  + s_{2i+1} s_{2i+2}  }     }  } }  }} } }  \\
&=\sum_{s_{2i}=\pm 1}{  \prod_{2i}{  \mkl{2 \text{cosh}\ekl{K\kl{s_{2i}+s_{2i+2}}}} } }  \\
&=\sum_{s_{2i}=\pm 1}{  \prod_{2i}{  z\kl{K} e^{K' s_{2i}s_{2i+2} } } }  \\
&=\ekl{z\kl{K}}^{\frac{N}{2}}\sum_{s_{2i}=\pm 1}{  \prod_{2i}{   e^{K' s_{2i}s_{2i+2} } } }  
\end{align*}
There are only two possible states, and we can compute $z\kl{K}e^{K's_{2i}s_{2i+2}}= 2\text{cosh}\ekl{K\kl{ s_{2i}+s_{2i+2} }} $ explicitly:

\begin{equation}
z\kl{K}e^{K's_{2i}s_{2i+2}} =\begin{cases}
  2\text{cosh}\ekl{2K}  & \text{for }s_{2i}=s_{2i+2}\\
  2  & \text{for }s_{2i}=-s_{2i+2}
\end{cases}
\label{eq:condition_renorm}
\end{equation}
By dividing and multiplying the two cases we get
$$
e^{2K'} =  \text{cosh}\ekl{2K}  \text{ and } z^2\kl{K} =  4\text{cosh}\ekl{2K} 
$$
$$
\Rightarrow  K' =  \frac{1}{2}\text{ln}\kl{   \text{cosh}\ekl{2K}   }. 
$$
We have now obtained a rule after which the coupling constant $K$ (which contains the temperature and the spin coupling constant $J$) changes under scaling. In summary, we:
\begin{itemize}
\item chose the scaling rule (decimation).
\item imposed the free energy to be constant
\item computed the consequences on the coupling constant $J$
\end{itemize}



 Generally, when renormalizing a system, there are two steps that one has to undertake:
\begin{itemize}
\item Decide which scale to change (e.g., the length of the system, the spins, etc.).
\item Evaluate the consequences for the system (e.g., for the Hamiltonian, the free energy, etc.).
\end{itemize}




\subsection{Generalization}
\label{subsec:generalization}

In the above example it was possible to fulfill the condition of constant free energy with just one coupling constant. Luckily, the transformation rule for the coupling constant $K$ was enough to fulfill all the conditions expressed by eq. \eqref{eq:condition_renorm}. Generally this is not the case, and more coupling constants (e.g., next nearest neighbors) are needed to close the system of equations. We therefore have to construct a renormalized Hamiltonian made up of renormalized coupling constants which generally contain many possible interactions:


\begin{equation}
\tilde{H} = \sum_{\alpha=1}^M{\tilde{K_{\alpha}} \tilde{O}_{\alpha}} 
\text{ with }
\tilde{O}_{\alpha} =  \sum_{i}{\prod_{k\in \epsilon_\alpha}{\tilde{\sigma}_{i+k}}} 
\end{equation}
and with the renormalization rules
$$
\tilde{K}_{\alpha}\kl{K_1,...,K_M},  \text{ with } \alpha \in \mkl{1,...,M}.
$$
Note that using only $M$ interaction terms  instead of an infinite number is a truncation, and in fact a systematic error. The accuracy of this method will depend on the number of iterations that we want to take into account. At $T_c$ we have a fixed point: $\tilde{K}_{\alpha}\kl{K_1^*,...,K_M^*}=K^*_\alpha$. A first ansatz to solve this problem is the linearization of the transformation. We do this calculating the Jacobian $T_{\alpha,\beta} = \pder{\tilde{K}_\alpha}{K_\beta}$:

\begin{equation}
 \tilde{K}_\alpha - K_\alpha^* = \sum_\beta { \left .T_{\alpha, \beta}\right|_{K^*}\kl{K_\beta - K_\beta^*}}
 \label{eq:lin_trafo}
\end{equation}


\vspace{0.1cm}
\noindent
\begin{minipage}{\textwidth}
\begin{minipage}{.48\textwidth}
 We can now construct a flow chart of the coupling constant and obtain values for $\tilde{K}$ for each vector $K=\kl{K_1,...,K_M}$. At $T_c$ (where $K=K^*$), we will have a fix point. Fig. \ref{fig:renorm_flow} shows a flow diagram of a renormalization group transformation. It can already be seen that the flow is stable in some directions (the ones in which the system tends to the fixed point) and unstable in others (the ones that lead the system to go away from the fixed point). 
 \end{minipage}\hfill
\begin{minipage}{.48\textwidth}
  \centering
  \includegraphics[height=160pt]{pics/renorm_flow}
  \captionof{figure}{Flow diagram of a renormalization transformation.}
  \label{fig:renorm_flow}
\end{minipage}
\end{minipage}

\vspace{0.2cm}


To analyze the behavior of the system close to criticality, we will consider the eigenvalues $\lambda_1,...,\lambda_M$ and the eigenvectors $\phi_1,...,\phi_M$ of the linearized transformation \ref{eq:lin_trafo}. From linear algebra we know that $\tilde{\phi_\alpha}=\lambda_\alpha\phi_\alpha$. Clearly, if $\lambda_\alpha>1$ we have an unstable situation (the distance from $\vec{K}*$ increases at every iteration of the transformation). The biggest eigenvalue will be the dominating one (which is the temperature exponent in the Ising model). We can identify the scaling field $\tilde{\epsilon}=l^{y_T}\epsilon$ with the eigenvector of the transformation, and the scaling factor with eigenvalue $\lambda_T = l^{y_T}$,

$$
\Rightarrow \nu = \frac{1}{y_T}=\frac{\text{ln}\ekl{l}}{\text{ln}\ekl{\lambda_T}}.
$$
This means that we can now calculate critical exponents using the scaling behavior of the system, if the scaling factor $l$ is known. 





\subsection{Monte Carlo Renormalization Group}
\label{subsec:MCRG}

The implementation of real space renormalization with Monte Carlo techniques was first proposed by  \citet{ma_renorm} and then reformulated by \citet{swendsen_renorm}. Since we are dealing with generalized Hamiltonians with a lot of interaction terms, we will calculate the thermal average using the operators $O_\alpha$.

\begin{equation}
\avkl{O_{\alpha}} =
\frac    {\sum_\alpha {O_\alpha e^{\sum_\beta {K_\beta O_\beta}} }  }   {\sum_\alpha {e^{\sum_\beta {K_\beta O_\beta}}}  } =
\pder {F}{K_\alpha}
\label{eq:op_av}
\end{equation}
where $F=\text{ln}\ekl{Z}$ is the free energy. Using the fluctuation-dissipation theorem one can also numerically calculate the response functions:
\begin{align*}
\chi_{\alpha,\beta} &\equiv \pder{\avkl{O_\alpha}}{K_\beta} = \avkl{O_\alpha O_\beta} -\avkl{O_\alpha }\avkl{ O_\beta}\\
\tilde{\chi}_{\alpha,\beta} &\equiv \pder{\avkl{\tilde{O}_\alpha}}{K_\beta} = \avkl{\tilde{O}_\alpha O_\beta} -\avkl{\tilde{O}_\alpha }\avkl{ O_\beta}
\end{align*}
Using the chain rule,  one can calculate with equation \eqref{eq:op_av} that



$$
\tilde{\chi}_{\alpha,\beta} \equiv \pder{\avkl{\tilde{O}_\alpha}}{K_\beta} =
\sum_\gamma {     \pder{\tilde{K}_\gamma}{K_\beta}    \pder{\avkl{\tilde{O}_\alpha^{\kl{n}}}}{K_\gamma}    }  =
\sum_\gamma  {   T_{\gamma, \beta}  \chi_{\alpha,\gamma}^{\kl{n+1}}  }.
$$
Thus we can obtain $T_{\gamma, \beta}$ from the correlation functions by solving a set of M coupled linear equations. We can iterate this method, in order to get systematically better values, if we are at the point $K=K^*$.

\vspace{0.1cm}
\noindent
\begin{minipage}{\textwidth}
\begin{minipage}{.98\textwidth}
  \centering
  \includegraphics[height=210pt]{pics/mc_renorm}
  \captionof{figure}{Scheme of the MCRG strategy.}
  \label{fig:mc_renorm}
\end{minipage}
\end{minipage}
\vspace{0.1cm}



\vspace{0.1cm}
\noindent
\begin{minipage}{\textwidth}
\begin{minipage}{.98\textwidth}
  \centering
  \includegraphics[height=170pt]{pics/nu4_potts}
  \captionof{figure}{$\nu$ for a 4-state Potts model in 2D. Note the convergence.}
  \label{fig:nu4_potts}
\end{minipage}
\end{minipage}
\vspace{0.1cm}



\subsubsection*{Errors in MCRG}

There are many error sources in this technique, that originate from the fact that we are using a combination of several tricks to calculate our results, of which one should be aware of:

\begin{itemize}
\item Statistical errors
\item Truncation of the Hamiltonian to the M$^{th}$ order.
\item Finite number of scaling iterations
\item Finite Size effects
\item No precise knowledge of $K^*$
\end{itemize}


















%------------------------------------------------------------------------------------------------------------------------------------------
\include{chapters/parallelization/multi_spin}
\include{chapters/parallelization/vectorization}
%------------------------------------------------------------------------------------------------------------------------------------------
\include{chapters/molecular_dynamics/introduction}
\include{chapters/molecular_dynamics/verlet_leap}

\section{Dynamics of Composed Particles}


In nature it is easy to find systems in which the interactions also depends on the size and the shape of the particles (e.g. molecules, crystals, landslides...). To approximate the motion, the interaction and the development of such systems one has to consider the shape and the composition of its composing particles. We will therefore start with the model of rigid bodies, and then relax the condition of rigidity a bit. This implies the following important consideration: it would be an oversight if we would simulate at energies at which the bonds that compose the particles are destroyed. 


\vspace{0.1cm}
\noindent
\begin{minipage}{\textwidth}
\begin{minipage}{.6\textwidth}%
Given the assumption that the bonds are stable in the simulated energy regime, there is a wide range of situations in which these methods are very useful. As an example, at room temperature the  air molecules are not going to break up in their components. The atoms that compose the molecules will also not break up if not in very special situations (ionization in the higher atmosphere or on stars' surfaces, particle accelerators, etc.). As a practical example, consider a water molecule ($H_2O$). In this case, the distance and angles between the atoms are fixed.
 \end{minipage}%
\hfill
\begin{minipage}{.4\textwidth}%
  \centering
  \includegraphics[width=\textwidth]{pics/water}
  %\captionof{figure}{Comparison of the precision in the Verlet method and the predictor-corrector of various order.}
  \label{fig:water}
\end{minipage}
\end{minipage}
\vspace{0.1cm}

There are two main methods applied to such a situation:
\begin{itemize}
\item Lagrange multipliers 
\item Rigid body approximation (good for arbitrary shapes)
\end{itemize}



\subsection{Lagrange multipliers}

\citet{lagrange} were one of the first to propose an extra force term in the equations of motion to impose the constraints given by a speficic shape of the composed particle (e.g. a molecule):

\begin{equation}
m_i\ddot{\vec{x}}_i = \underbrace{f_i}_{\text{external interaction}} + \underbrace{\vec{g}_i}_{\text{internal constraints}}
\end{equation}

We can impose constraining forces that will enforce the geometric arrangement of the molecules, e.g. certain distances $d_{1,2}$ and $d_{2,3}$ between atoms. In this case, the constraint forces are proportional to the difference of the actual and the desired distance of the particles. We can define a potential:

$$
\chi_{1,2} \equiv r^2_{1,2} -d^2_{1,2} \overset{\text{rest}}{=} 0
$$
\begin{equation}
\chi_{2,3} \equiv r^2_{2,3} -d^2_{2,3} \overset{\text{rest}}{=} 0
\end{equation}

The equality to zero only holds if the particles have a distance ($r_{ij}$)  equal to the chosen rest position $d_{ij}$. With $r_{ij}\equiv \abs{\vec{r}_{i,j}}$, $\vec{r}_{i,j} \equiv \vec{x}_{i}  -\vec{x}_{j} $ one can obtain
  $$ \vec{g}_k = \frac{\lambda_{1,2} }{2} \vec{\nabla} _{\vec{x}_k} \chi_{1,2}+
                   \frac{\lambda_{2,3} }{2} \vec{\nabla} _{\vec{x}_k} \chi_{2,3}
 $$ we can define the Lagrange multipliers, $\lambda_{1,2}$ and $\lambda_{2,3}$, yet to be determined. We will compute these multipliers by imposing constraints. The force is then obtained by the gradient of the potential:
 \begin{equation}
 \Rightarrow \vec{g}_1 = \lambda_{1,2} \vec{r}_{1,2}, \hspace{0.4cm} 
\vec{g}_2 = \lambda_{2,3} \vec{r}_{2,3} - \lambda_{1,2} \vec{r}_{1,2}, \hspace{0.4cm} 
\vec{g}_3 = - \lambda_{2,3} \vec{r}_{2,3}.
\label{eq:lagrange_forces}
\end{equation}
 This is simply a linear spring with a  yet to be determined spring constant $\lambda$. To obtain the values of $\lambda$, the Verlet algorithm is executed in two steps: first we will compute the forces on the molecules:
 $$
\tilde{\vec{x}}_i \kl{t+\Delta t} = 2\vec{x}_i  - \vec{x}_i\kl{t-\Delta t} + \Delta t ^2 \frac{f_i}{m_i} 
$$ 
and then we correct the value using the above the constraints:
$$
\vec{x}_i \kl{t+\Delta t} = \tilde{\vec{x}}_i \kl{t+\Delta t} + \Delta t ^2 \frac{\vec{g}_i}{m_i} 
$$
We can insert \eqref{eq:lagrange_forces} into the last expression to find the updated positions:

\begin{align}
\vec{x}_1 \kl{t+\Delta t} &= \tilde{\vec{x}}_1 \kl{t+\Delta t} + \Delta t ^2 \frac{\lambda_{1,2}}{m_2} \vec{r}_{2,3}\kl{t}\\
\vec{x}_2 \kl{t+\Delta t} &= \tilde{\vec{x}}_2 \kl{t+\Delta t} + \Delta t ^2 \frac{\lambda_{2,3}}{m_1} \vec{r}_{1,2}\kl{t} -  \Delta t ^2 \frac{\lambda_{1,2}}{m_2} \vec{r}_{1,2}\kl{t}\\
\vec{x}_3 \kl{t+\Delta t} &= \tilde{\vec{x}}_3 \kl{t+\Delta t} + \Delta t ^2 \frac{\lambda_{2,3}}{m_3} \vec{r}_{2,3}\kl{t}
\end{align}
With these expressions, we can obtain now $\lambda_{1,2}$ and $\lambda_{2,3}$ by inserting them into the constraint condition:

$$
\abs{\vec{x}_1 \kl{t+\Delta t} - \vec{x}_2 \kl{t+\Delta t}}^2 = d_{1,2}
$$
$$
\abs{\vec{x}_2 \kl{t+\Delta t} - \vec{x}_3 \kl{t+\Delta t}}^2 = d_{2,3}
$$
$$
\Rightarrow
$$

$$
\abs{      \tilde{ \vec{x}}_1 \kl{t+\Delta t} -\tilde{  \vec{x}}_2 \kl{t+\Delta t} + \Delta t^2 \lambda_{1,2} \kl{\frac{1}{m_1} +\frac{1}{m_2}}\vec{r}_{1,2}\kl{t}  - \Delta t^2 \frac{\lambda_{2,3}}{m_2 \vec{r}_{2,3}\kl{t}}        }^2 = d_{1,2}
$$
$$
\abs{      \tilde{  \vec{x}}_2 \kl{t+\Delta t} - \tilde{ \vec{x}}_3 \kl{t+\Delta t} + \Delta t^2 \lambda_{2,3} \kl{\frac{1}{m_2} +\frac{1}{m_3}}\vec{r}_{2,3}\kl{t}  - \Delta t^2 \frac{\lambda_{2,3}}{m_2 \vec{r}_{1,2}\kl{t}}        }^2 = d_{2,3}
$$
These expressions can be solved  for $\lambda_{1,2}$ and $\lambda_{2,3}$ that can be then used to calculate the next position $\vec{x}_i\kl{t+\Delta t}$. Depending on the precision needed one can ignore the higher order terms of $\Delta t$.


\subsection{Rigid Bodies}
In the case of a rigid body (an object in which the distances of the particles composing it remain constant), particle motion can be split into translation of the center of mass and rotation of the body: the center of mass
\begin{equation}
M\vec{x}_{cm} \equiv \sum_{i=1}^n{\vec{x}_im_i}
\hspace{0.4cm}\text{with}\hspace{0.4cm} 
M\equiv \sum_{i=1}^n{m_i}
\label{eq:center_of_mass}
\end{equation}

follows the equations of motion and the rotation is given by the torque,
\begin{equation}
M\ddot{\vec{x}}_{cm} = \sum_{i=1}^n{f_i} \equiv f_{com}
\hspace{0.4cm}\text{and}\hspace{0.4cm} 
\vec{T} \equiv \sum_{i=1}^n{     \vec{d}_i \times     f_i   }
\label{eq:cof_eom}
\end{equation}
with $\vec{d}_i \equiv \vec{x}_i - \vec{x}_{cm}$.

In two dimensions  the rotation always points in the direction of the normal vector of the plane. There are therefore only three degrees of freedom: two translational and one rotational. In three dimensions there are six degrees of freedom. We well first introduce the two-dimensional case and then generalize to the three-dimensional case.

\subsubsection*{2D:}
In 2D,  the moment of inertia and the torque are given by

$$
I = \int\int_{A}{r^2 \rho \kl{r} \text{dA}}
\hspace{0.4cm}\text{and}\hspace{0.4cm} 
T = \int\int_{A}{f_t\kl{r} r \text{dA}}.
\label{eq:cof_eom}
$$
From Newton's equations one can derive the equation of motion for the rotation:
\begin{equation}
I\dot{\omega} = T.
\label{eq:eom_rot}
\end{equation}

We can calculate the time evolution of the rotation angle $\phi$ by applying the Verlet algorithm to the variable $\phi\kl{t}$:
$$
\phi\kl{t+\Delta t} = 2 \phi\kl{t} -\phi\kl{t-\Delta t} + \Delta t^2 \underbrace{ \frac{T\kl{t}}{I}}_{=\dot{\vec{\omega}}}
$$
 $$
\vec{x}\kl{t+\Delta t} = 2 \vec{x}\kl{t} -\vec{x}\kl{t-\Delta t} + \Delta t^2 M^{-1} \sum_{j\in A} {f_j\kl{t}}
 $$
 where the torque can be calculated summing over all the torques in the body: $T\kl{t} = \sum_{j\in A} \ekl{    f^y_j\kl{t} d^x_j\kl{t} -  f^x_j\kl{t} d^y_j\kl{t}     }$. 
 

\subsubsection*{3D:}
If we expand our model to the third dimension, we will see that the computation is not as simple as on a plane where the torques and angular momenta always point in the same direction. As in classical mechanics we define the angular momentum as
$$
\vec{l} \equiv \sum_{i=1}^n{m_i\vec{d}_i \times \vec{v}_i} 
= \sum_{i=1}^n{m_i\vec{d}_i \times \kl{\vec{d}_i \times \vec{\omega}}} 
= \sum_{i=1}^n {m_i   \kl{\vec{d}_i    \kl{\vec{d}_i \vec{\omega}}   -\vec{d}_i^{\,\,2}\vec{\omega} }  }     
= \mat{I} \vec{\omega}.
$$
With this definition the equation of motion is
$$
\dot{\vec{l}} = \mat{I}\dot{\vec{\omega}} = \vec{T}
$$  
where $\mat{I}$ is the \emph{tensor of inertia}. The tensor of inertia describes the motion of rigid bodies and it can be generally written as
$$
\mat{I} = \sum _{i=1}^n {m_i \kl{  \vec{d}_i^{\,\,T}  \bigotimes    \vec{d}_i   -\vec{d}_i^{\,\,2} \vec{l}\,\,    }       }
$$
 This tensor can be brought into a diagonal form by transforming the coordinate system into a body-fixed coordinate system with origin in the center of mass and basis vectors pointing in the directions of the eigenvectors of the tensor. The basis transformation can be written with a matrix $\mat{A}$:
 \begin{equation}
\underbrace{ \vec{e}^{\,\,b} }_{b.c.s.}= \underbrace{ \mat{A}\vec{e}^{\,l} }_{l.c.s.}
\label{eq:transf_mat}
\end{equation}

 $$
\dot{\vec{l}}^{\,\,l} = \vec{T}^{\,l} 
\hspace{0.4cm} \Rightarrow \hspace{0.4cm}
\dot{\vec{l}}^{\,\,b} + \vec{\omega}^b\times\vec{l}^{\,\,b} = \mat{I} \dot{\vec{\omega}}^b + \vec{\omega}^b \times \vec{l}^{\,\,b} = \vec{T}^{\,b}
\hspace{0.4cm} \Leftrightarrow \hspace{0.4cm}
\mat{I} \dot{\vec{\omega}}^b  = \vec{T}^{\,b} - \vec{\omega}^b \times \vec{l}^{\,\,b}
$$

Without derivation, this can be then transformed into a system of equations

\begin{align}
\dot{\vec{\omega}}_x^b &= \frac{\vec{T}^b_x}{I_{xx}} + \kl{\frac{\mat{I}_{yy}-\mat{I}_{zz}}{\mat{I}_{xx}}   } \vec{\omega}^b_y\vec{\omega}^b_z \\
\dot{\vec{\omega}}_y^b &= \frac{\vec{T}^b_y}{I_{yy}} + \kl{\frac{\mat{I}_{zz}-\mat{I}_{xx}}{\mat{I}_{yy}}   } \vec{\omega}^b_z\vec{\omega}^b_x \\
\dot{\vec{\omega}}_z^b &= \frac{\vec{T}^b_z}{I_{zz}} + \kl{  \frac{\mat{I}_{xx}-\mat{I}_{yy}}{\mat{I}_{zz}}   } \vec{\omega}^b_x\vec{\omega}^b_y
\end{align}
with the tensor of inertia being diagonal in the body frame: 
$$\mat{I} = \begin{pmatrix}
 I_{xx} & 0 & 0\\
 0 & I_{yy} & 0 \\
 0 & 0 & I_{zz}  
\end{pmatrix}. $$ 
As we can see, the diagonal form of the tensor of inertia comes with the cost of one added term. Together with \eqref{eq:transf_mat} we can compute the angular velocities:

$$
\vec{T}^{\,l} = \sum_{i=1}^n   {\vec{d}_i \times f_i} 
\hspace{0.4cm} \Rightarrow \hspace{0.4cm}
\vec{T}^b = \mat{A}\vec{T}^l
$$


\begin{align}
\vec{\omega}^b_x \kl{t+\Delta t}  &= \vec{\omega}^b_x \kl{t} + 
\Delta t \frac  {\vec{T}^b_x \kl{t} }{I_{xx}} + \Delta t \kl{\frac{\mat{I}_{yy}-\mat{I}_{zz}}{\mat{I}_{xx}}   } \vec{\omega}^b_y\vec{\omega}^b_z \\
\vec{\omega}^b_y \kl{t+\Delta t} &= \vec{\omega}^b_y \kl{t} + 
\Delta t \frac  {\vec{T}^b_y \kl{t} }{I_{yy}} + \Delta t \kl{\frac{\mat{I}_{zz}-\mat{I}_{xx}}{\mat{I}_{yy}}   } \vec{\omega}^b_z\vec{\omega}^b_x \\
\vec{\omega}^b_z \kl{t+\Delta t} &= \vec{\omega}^b_z \kl{t} + 
\Delta t \frac  {\vec{T}^b_z \kl{t} }{I_{zz}} + \Delta t \kl{\frac{\mat{I}_{xx}-\mat{I}_{yy}}{\mat{I}_{zz}}   } \vec{\omega}^b_x\vec{\omega}^b_y
\end{align}
From these expressions  and \eqref{eq:transf_mat}  one can easily obtain the angular velocity in the laboratory frame:
$$
\vec{\omega}^l \kl{t+\Delta t} = \mat{A}^T \vec{\omega}^b\kl{t+\Delta t}.
$$
Since the particles are moving all the time, the transformation matrix in equation \eqref{eq:transf_mat} is not constant. We therefore have to find an efficient way to determine and update $\mat{A}$ at every step. This can be done using either Euler angles and quaternions. The following will be only a summary of the derivation, since these topics are generally treated in classical mechanics.



\subsubsection*{Euler angles:}
We will begin by defining the transformation matrix for a rotated frame of reference. We assume that the transformation matrix for a rotation around the Euler angles is already known. There are a huge number of different conventions. One way to represent an arbitrary rotation is through the following combination of rotations:
$$
\mat{A} = 
\begin{pmatrix}
 \cos{\Psi}  & -\sin{\Psi} & 0 \\
 \sin{\Psi} & \cos{\Psi} &0\\
 0 & 0 & 1   
\end{pmatrix} 
\cdot
\begin{pmatrix}
 1 & 0 & 0 \\
0 & \cos{\Theta} & -\sin{\Theta} \\
 0 & \sin{\Theta} & \cos{\Theta}  
 \end{pmatrix} 
\cdot
\begin{pmatrix}
 \cos{\Phi} & -\sin{\Phi} & 0  \\
 \sin{\Phi} & \cos{\Phi}  & 0  \\
 0 & 0 & 1  
\end{pmatrix} 
$$
\begin{equation}
.
 \label{eq:horr_mat}
\end{equation}

$$
= 
\begin{pmatrix}
 \cos{\Phi} \cos{\Psi}  -\sin{\Phi}\sin{\Psi}-\cos{\Theta}   & \sin{\Phi}\cos{\Psi} + \cos{\Phi}\cos{\Theta}\sin{\Psi} & \sin{\Theta}\sin{\Psi}\\
 -\cos{\Phi}\sin{\Psi} +\sin{\Phi}\cos{\Theta}\cos{\Psi} & -\sin{\Phi}\sin{\Psi} + \cos{\Phi}\cos{\Theta}\cos{\Psi} & \sin{\Theta}\cos{\Psi} \\
 \sin{\Phi}\sin{\Theta} & -\cos{\Phi}\sin{\Theta} & \cos{\Theta}
% \label{eq:horr_mat}
\end{pmatrix} .
$$

The take-home message here is that an arbitrary rotation can assume a very nasty form which is everything but well suited for an efficient implementation. One has to keep in mind that this operation has to be done for every particle and for every time step, making this approach prohibitive for what concerns the time consumption. One can also find analytical relations to the angular velocities
\begin{align}
\dot{\Phi} &= -\omega^l_x \frac{\sin{\Phi}\cos{\Theta}}{\sin{\Theta}} + \omega^l_y \frac{\cos{\Phi}\cos{\Theta}}{\sin{\Theta}} + \omega^l_z  \\
\dot{\Theta} &= -\omega^l_x\cos{\Theta} + \omega^l_y \sin{\Phi}  \\
\dot{\Phi} &= \omega^l_x \frac{\sin{\Phi}}{\sin{\Theta}} - \omega^l_y \frac{\cos{\Phi}}{\sin{\Theta}} 
\label{eq:ang_veloc}
\end{align}
which again are not very suitable for computation. Furthermore, one can see that this representation also has singularities, since we are dividing by $\sin{\Theta}$. We therefore have to find alternative expressions for the rotational motion.

\subsubsection*{Quaternions:}
Daniel Evans, a professor in Canberra, Australia, came up with a trick to optimize the computation of rotational velocities \citep{evans1,evans2}. This is a rather algebraic approach, which is not at all intuitive, and we will only discuss the main steps of the method here. 

Quaternions are a generalization of complex numbers, where four basis vectors span a four-dimensional space. By defining

\begin{align}
q_0 &\equiv \text{cos}\ekl{\frac{\Theta}{2}}  \text{cos}\ekl{\frac{\Phi+\Psi}{2}} \\
q_1 &\equiv \text{sin}\ekl{\frac{\Theta}{2}}  \text{cos}\ekl{\frac{\Phi-\Psi}{2}} \\
q_2 &\equiv \text{sin}\ekl{\frac{\Theta}{2}}  \text{sin}\ekl{\frac{\Phi-\Psi}{2}} \\
q_3 &\equiv \text{cos}\ekl{\frac{\Theta}{2}}  \text{sin}\ekl{\frac{\Phi+\Psi}{2}} 
\label{eq:quaternions}
\end{align}
we can represent the angles in dependence of our set of quaternions $q_i$. Note that the euclidean norm of $\vec{q}$ equals unity, therefore there are in fact only three degrees of freedom. Skipping the derivation, one can show that


$$
\mat{A}= 
\begin{pmatrix}
 q_0^2 + q_1^2 -q_2^2 -q_3^2  & 2\kl{q_1q_2+q_0q_3}  & 2\kl{q_1q_3-q_0q_2} \\
 2\kl{q_1q_2-q_0q_3}  & q_0^2 -q_1^2 +q_2^2 -q_3^2 &  2\kl{q_2q_3+q_0q_1} \\
  2\kl{q_1q_3+q_0q_2} &  2\kl{q_2q_3+q_0q_1}  & q_0^2 -q_1^2 -q_2^2 +q_3^2
\end{pmatrix} 
$$
We now have a closed form to represent our rotation, without having to calculate any products involving $\sin{x}$ and $\cos{x}$. This is by order of magnitudes faster than the expression in \eqref{eq:horr_mat}. It is a matter of algebra showing that


$$
\begin{pmatrix}
 \dot{q}_0\\
 \dot{q}_1\\
  \dot{q}_2\\
   \dot{q}_3
   \end{pmatrix} 
= \frac{1}{2}
\begin{pmatrix}
 q_0 & - q_1 &  -q_2 & -q_3 \\
 q_1 &   q_0 &  -q_3 &  q_2 \\
 q_2 &   q_3 &   q_0 & -q_1 \\
 q_3 & - q_2 &   q_1 &  q_0 
\end{pmatrix} 
\cdot
\begin{pmatrix}
 0\\
\omega^b_x\\
\omega^b_y\\
\omega^b_z\\
   \end{pmatrix} 
$$
%Since the world of quaternions and the normal euclidean space are connected by a diffeomorphism, there is always the possibility of calculating the values of the Euler angles if needed (e.g. for a plot):
If needed (e.g. for a plot) the Euler angles can be calculated from quaternions:

\begin{align}
\Phi &= \text{arctan}\ekl{ \frac{2\kl{q_0q_1+q_2q_3}}{ 1- 2\kl{q_1^2+q_2^2}}  }  \\
\Theta &= \text{arcsin}\ekl{ 2\kl{q_0q_2-q_1q_3} }  \\
\Psi &= \text{arctan}\ekl{ \frac{2\kl{q_0q_3+q_1q_2}}{ 1- 2\kl{q_2^2+q_3^2}}  }  
\label{eq:quaternions_back}
\end{align}

Mind that there is no need of calculating the Euler angles at each integration step. We can run our simulation completely with the sole use of the quaternion representation of our rigid body motion. Thus the strategy to follow is

\begin{itemize}
\item Calculate torque $T\kl{t}$ in the body frame.
\item Obtain $\omega^b\kl{t+\Delta t}$ (in quaternion representation).
\item Integrate the equation of motions (remaining in the quaternion representation).
\end{itemize}












\section{Simulating Shapes}


A very important issue in computational physics is dealing with particles and objects of arbitrary shapes interacting with each other. There are many possible solutions for this problem, depending on the shapes that one wants to simulate. In some very special cases, like ellipsoidal particles, it is possible to find analytic solutions for the equations of motion.



\subsection{Ellipsoidal particles}

The general parametrisation of two ellipses (see Fig. \ref{fig:ellipses}) in 2D is given by the equations

\begin{align*}
\kl{\frac{x-x_a}{a_1}}^2 + \kl{\frac{y-y_a}{a_2}}^2 &= 1\\
\kl{\frac{x-x_b}{b_1}}^2 + \kl{\frac{y-y_b}{b_2}}^2 &= 1
\end{align*}


Perram and Wertheim \citep{perram} proposed to take the overlap of the shapes as a measure for the interaction. To calculate the overlap one can transform the ellipses into circles using an appropriate metric \citep{comp_phys}. We can generalize an ellipse with the functional 
\begin{equation}
G_A\kl{\vec{r}} = \kl{\vec{r}-\vec{r}_a}^{\,T} \mat{A} \kl{\vec{r}-\vec{r}_a}
\label{eq:ellipse_func}
\end{equation}

which is smaller than one inside the ellipse, larger then one outside and exactly one on the ellipse:

\begin{equation}
G_A\kl{\vec{r}} =\begin{cases}
  <1,  & \text{if } \vec{r} \text{ is inside the ellipse}\\
  \,\,\,\,\,\,1,  & \text{if } \vec{r} \text{ is on the ellipse}\\
  >1,  & \text{if } \vec{r} \text{ is outside the ellipse}
\end{cases}
\end{equation}



\vspace{0.1cm}
\noindent
\begin{minipage}{\textwidth}
\begin{minipage}{.85\textwidth}
  \centering
  \includegraphics[width=.85\textwidth]{pics/ellipses.jpg}
  \captionof{figure}{Parameters for the characterization of two ellipses.}
  \label{fig:ellipses}
\end{minipage}
\hfill
\begin{minipage}{.001\textwidth}
 \end{minipage}
\end{minipage}
\vspace{0.1cm}

We can now define a new joint functional for the two ellipses that is weighted with a parameter $\lambda$ that interpolates between the two centers of the ellipses defined with \eqref{eq:ellipse_func}. $\lambda$ goes from 0 to 1, and in the extrema the functional signs the centre of the first or the second ellipse respectively.

\begin{equation}
G\kl{\vec{r},\lambda} = \lambda G_A\kl{\vec{r}}  + \kl{1-\lambda} G_B\kl{\vec{r}} 
\end{equation}

For every $\lambda$ we have a surface, and for every $\lambda$ we are interested in the minimum of the surface. To find the minimum we have to minimize the functional:
\begin{equation*}
\nabla_{\vec{r}}\,G\kl{\vec{r},\lambda} = 0.
\end{equation*}
which yields
\begin{equation}
\vec{r}_m \kl{\lambda} = \kl{\lambda \mat{A} + \kl{1-\lambda} \mat{B}}^{-1} \kl{\lambda\mat{A}\vec{r}_a + \kl{1-\lambda}\mat{B}\vec{r}_b}.
\end{equation}
If we start from $\lambda=0$ and arrive to $\lambda=1$ we will get a path from the center of the first ellipse to the other. We can rewrite the path defining $\vec{r}_{ab}\equiv \vec{r}_b-\vec{r}_a$ and obtain

$$
\vec{r}_m \kl{\lambda} = \vec{r}_a +\kl{1-\lambda} \mat{A}^{-1} \ekl{  \kl{1-\lambda} \mat{A}^{-1} + \lambda \mat{B}^{-1}  }^{-1} \vec{r}_{ab}
$$
\begin{equation}
\vec{r}_m \kl{\lambda} = \vec{r}_b -\kl{1-\lambda} \mat{B}^{-1} \ekl{  \kl{1-\lambda} \mat{A}^{-1} + \lambda \mat{B}^{-1}  }^{-1} \vec{r}_{ab}
\label{eq:min_path}
\end{equation}

These paths are very handy: if the value of the functional along the path between the centers always smaller than 1, we know that we did not have to leave the ellipses, hence they overlap. We can define an \emph{overlap function} 
\begin{equation}
S\kl{\lambda} \equiv G\kl{\vec{r\kl{\lambda}},\lambda} 
\end{equation}
and if we insert \eqref{eq:min_path} we get
\begin{equation}
S\kl{\lambda} = \lambda \kl{1-\lambda} \vec{r}_{ab}^{\,T}\ekl{  \kl{1-\lambda}\mat{A}^{-1}+ \lambda \mat{B}^{-1}   }^{-1}\vec{r}_{ab}
\end{equation}
This now is the height of the minimal path that connects the two centers. As already mentioned, we are interested if the path ever assumes values larger than 1. If we maximize $S$, we will be able to tell if the ellipses are overlapping or not:


\begin{equation}
S\kl{\lambda_{\text{max}}} =\begin{cases}
  <1,  & \text{if the ellipses overlap}\\
  \,\,\,\,\,\,1,  & \text{if the ellipses touch}\\
  <1,  & \text{if the ellipses are separated}
\end{cases}
\end{equation}

With this knowledge we can also calculate the contact point, if the ellipses are stiff and we have to avoid the overlap through an (in)elastic collision. We set $S\kl{\lambda_{\text{max}}}$ to unity and find the contact point using \eqref{eq:min_path}. 


Ellipses can be generalized to so-called \emph{superellipsoids} \citep{superellipsoids} and the macroscopic effects of these generalizations are all but trivial \citep{mms}. Industrial engineering, fluiddynamics in biology (e.g. blood cells) and many other fields often rely on simulations of macroscopic particles, often ellipsoidal, and this is why this techniques have such a big resonance today.


We saw that even for the simplest generalization of spheres, it was already necessary to develop complex analytical methods. This significantly increase the computational complexity and the time consumption of the programs. For this reason, it is necessary to develop new approaches to handle objects of arbitrary shapes.
%13b -21min



\subsection{Polygons}


A particular class of macroscopic particles are those that can be described by polygons (e.g. rocks, sand grains, etc.).  In this case, a better measure for the repulsive forces is the overlap area (\emph{Cauchy elasticity}). The advantage of simple polygons is that the overlap area can be computed using simple geometries (e.g. dividing the area into triangles). However, between polygons there can be many different types of contacts, and in 3D the classification and the identification of all the types of contacts can be very cumbersome (see Fig. \ref{fig:poly_contact}).



\vspace{0.1cm}
\noindent
\begin{minipage}{\textwidth}
\begin{minipage}{.85\textwidth}
  \centering
  \includegraphics[width=.75\textwidth]{pics/poly_contact.jpeg}
 % \captionof{figure}{BLABLA.}
 % \label{fig:poly_contact}
\end{minipage}
\begin{minipage}{.85\textwidth}
  \centering
  \includegraphics[width=.75\textwidth]{pics/poly_contact2.jpeg}
  \captionof{figure}{Possible overlaps of 2D Polygons.}
  \label{fig:poly_contact}
\end{minipage}
\end{minipage}
\vspace{0.1cm}







\vspace{0.1cm}
\noindent
\begin{minipage}{\textwidth}
\begin{minipage}{0.35\textwidth}
Additional complexities arise when the overlap area does not represent the actual overlap of the polygons (see Fig. \ref{fig:poly_contact4}, upper row) or discontinuities can appear while particles move into another (see Fig. \ref{fig:poly_contact4}, lower row). Furthermore, rotation of particles need additional treatment, as the torque strongly depends on the shape of the particles.
\end{minipage}
\begin{minipage}{0.65\textwidth}
\begin{minipage}{.85\textwidth}
  \centering
  \includegraphics[width=.75\textwidth]{pics/poly_contact3.jpeg}
 % \captionof{figure}{BLABLA.}
 % \label{fig:poly_contact}
\end{minipage}
\begin{minipage}{.85\textwidth}
  \centering
  \includegraphics[width=.75\textwidth]{pics/poly_contact4.jpeg}
  \captionof{figure}{Possible issues with discontinuities.}
  \label{fig:poly_contact4}
\end{minipage}
\end{minipage}
\end{minipage}
\vspace{0.1cm}

Due to the mentioned reasons, the simulation of object composed of polygons can be tricky. In 3D, as always, things become horrible and the computation time for a large number of polyhedra increases enourmosly. 






\subsection{Spheropolygons}

% 13b -13


\vspace{0.2cm}
\noindent
\begin{minipage}{\textwidth}
\begin{minipage}{0.38\textwidth}
Another class of methods that are more efficient than the mere division into polygons  relies on \emph{spheropolygons} \citep{spheropolygons}. This technique was invented by Fernando Alonso-Marroquin, who uses the \emph{Minkowski cow} for demonstrative purpose. This particular shape is obtained by sweeping a disk around a polygon of multiple and arbitrary edges that resemble the shape of a cow (see Fig. \ref{fig:minkowski_cow}).
\end{minipage}
\hfill
\begin{minipage}{.58\textwidth}
  \centering
  \includegraphics[width=.75\textwidth]{pics/minkowski_cow.jpeg}
  \captionof{figure}{Decomposition of a Minkowski cow into spheres}
  \label{fig:minkowski_cow}
\end{minipage}
\end{minipage}
\vspace{0.1cm}

The take-home message is that the shape can be arbitrary. Once the shape is decomposed into spheres one only has to compute the contact between all the pairs of spheres that compose the edges and vertices of the shape which is easier than considering arbitrary shaped polygons.  There are of course a number of constraints when approximating the shapes in such a way. For example, too large spheres would smear out the original shape excessively and would even yield wrong results. Imagine that the spheres at the edges were larger than the distance between the hooves or the distance between the tale and the rest of the cow. Substantial features of the shape would be lost. For further videos and information, see \cite{spheropolygons2}.


\vspace{1.5cm}
Nowadays, There are many other techniques to describe arbitrary shaped objects. Many attempts to create more effective implementations are developed by engineers, phycisists and mathematicians. An important example is the field of \emph{mathematical morphology} \citep{serra} developed by Jean Serra in the 1960s. The techniques of Marroquin are related to the so called \emph{dilation techniques} but there are also other methods like the \emph{erosion technique}. We will not go deeper into the variety of existing methods since their mathematical frameworks reach a point in which technicalities monopolizes entire lectures and even courses, such as \emph{Computational Science and Engineering}.













\include{chapters/molecular_dynamics/long_range_potentials}


\section{Event Driven Programming}


In ``batch programming'', as the name suggests, a group of instructions is executed regardless of the physical situation and the events occurring during the simulations. As opposed to this, in an event driven program the simulation depends on what is actually happening in the simulated system\footnote{Note that this is not completely true, since it can also be the case for some particular routines based on batch programming, e.g. for adaptive time steps. This statement has to be taken as a loose classification of various programming philosophies rather than a strict rule.}. The flow is therefore interrupted by branching points and is ruled by conditional logic. These kinds of programs are generally difficult to parallelize due to their unpredictable nature. ``Unpredictable'', in the sense that the computations that the program executes strongly depend on the starting conditions of the problem and the evolution of the system. This is the reason why they typically are unpredictable. These general statement will become clearer with some concrete and intuitive examples.


\subsection{Elastic Collisions}

One of the first examples for event driven programming applied to molecular dynamics can be found in a publication by Alder  \citep{alder_spheres}. He considered rigid bodies of finite volume (like a set of billiard balls) which can be mathematically modeled by a hard core potential. This cannot be handled with the MD methods we learned, since the derivative of the potential diverges. In his simulation, Alder regarded the collisions between particles as instantaneous events without any further interaction between the particles but the collisions. This way, one can avoid the calculation of the forces, and only treat the momentum exchange during the collision. Mind that three body problems are neglected in the continuum mechanics of hard spheres and the calculations can therefore be solved analytically in some cases. We will start by describing the simplest of all cases: friction-free interaction of spheres with uniform density distribution.

Assuming that the collisions are instantaneous events with no influence on the dynamics before and after the impact, the system evolves undisturbed during the time in which no collision happens, $t_c$. The calculation of the uniform motion of the particles is relatively inexpensive, as it is the elastic collision of the particles involved in the impact. The expensive part of the program is the calculation of the time when the next collision occurs, since one has to go through all pairs of particles.

\vspace{0.1cm}
\noindent
\begin{minipage}{\textwidth}
\begin{minipage}{.001\textwidth}
 \end{minipage}\hfill
\begin{minipage}{.99\textwidth}
  \centering
  \includegraphics[width=.85\textwidth]{pics/2Dcollision}
  \captionof{figure}{Parameters for the collision event of two elastic disks.}
  \label{fig:2Dcollision}
\end{minipage}
\end{minipage}
\vspace{0.1cm}

In 2D, we can consider the collision of two disks $i$ and $j$. The \emph{collision angle} is defined as the angle betwee the connecting vector $\vec{r}_{i,j} \equiv \vec{r}_i-\vec{r}_j$ and the relative velocity $\vec{v}_{i,j} \equiv \vec{v}_i-\vec{v}_j$ (see Fig. \ref{fig:2Dcollision}). For the beginning, we will neglect the rotation since we are in a frictionless regime. The time $t_{i,j}$ at which the next collision between the particle $i$ and the particle $j$ would occur can be calculated using the formulae of uniform motion. From the vector connecting the centers of the particles
\begin{equation}
\abs{\vec{r}_{i,j}} \equiv \vec{r}_j-\vec{r}_i
\end{equation}
we can impose a certain length ($ \abs{R}_j+\abs{R}_i$) at the contact time and from their relative velocity ${v}_{i,j}$, the contact time of two particles can be calculated by solving 
\begin{equation}
{v}_{i,j}^2{t}_{i,j}^2 + 2\kl{\vec{r}_{i,j}^0 \vec{v}_{i,j}^2}{t}_{i,j} + \kl{{r}_{i,j}^0}^2 -\kl{ \vec{R}_i+\vec{R}_j}^2=0.
\label{eq:first_contact} 
\end{equation}
Mind that \eqref{eq:first_contact} only has valid solutions if the trajectories of the particles $i$ and $j$ do cross. The time $t_c$ when the next collision occurs in the system is then the minimum over all pairs $i,j$:

\begin{equation}
t_c=\underset{i,j}{\text{min}}\ekl{t_{i,j}}
\end{equation}

It is also possible to add simple external forces like the one induced by a gravitational or electromagnetic field. The main issue with this method is that the loop is very time consuming since it is of order $N^2$. For every step, one has to go through two minimization problems: first one has to find for each particle $i$ which will be the first particle $j$ that will collide with $i$ and calculate the collision time $t_{i,j}$.  If the system is very densely packed, there is no need to do the calculations for particle that are separated by large distances (large relatively to their mean free path). For this, the knowledge of the distance between each pair of particles is needed and the calculation would be very expensive. Instead of looking at the distances between particles one can divide the system into sectors and treat the sectors separately. The crossing of the particles through the sector boundaries has to be treated as a collision.
A further way of acelerating the algorithm is given by the \emph{Lubachevsky method}.




\subsection{Lubachevsky method}

Instead of going through all the pairs, one can create a list of events for each particle. In the list the events are stored keeping record of the time and particles involved in the previous and in the next  collision. After that an the particles are ordered depending on their next collision time. There will always be two particles for every collision time. When the collision occurs one has to reorganize  the system and the list of events only for the new particle. The reordering of the event list takes a time in the order of $O\kl{\text{log}N}$. The advantages of this method are that it is not necessary to minimize all the collision times of all the pairs at every step, and that it is unnecessary to propagate the particles that will not collide. Only the position and velocity of the particle involved in the event have to be updated. The rearrangement of the event list can be done via tournament trees instead of going through all the particles.



\subsubsection*{Collision with perfect slip:}

In a first approximation we will neglect that the particles can also exchange momentum tangentially. Angular momentum is therefore irrelevant, and only linear momentum is exchanged. We can use momentum 

\begin{align}
{\vec{v}_i}^{\,\text{after}} &= {\vec{v}_{i}}^{\,\text{before}} + \frac{\vec{\Delta p}}{m_i} \\
{\vec{v}_j}^{\,\text{after}} &= {\vec{v}_{j}}^{\,\text{before}} - \frac{\vec{\Delta p}}{m_j}
\end{align}
and energy conservation
\begin{equation}
\frac{1}{2}m_i \kl{\vec{v}_i^{\,\text{before}}}^2 + \frac{1}{2}m_j \kl{\vec{v}_j^{\,\text{before}}}^2 
=
\frac{1}{2}m_i \kl{\vec{v}_i^{\,\text{after}}}^2 + \frac{1}{2}m_j \kl{\vec{v}_j^{\,\text{after}}}^2 
\end{equation}
and we obtain the exchanged momentum: 
\begin{equation}
\vec{\Delta p} = -2 m_{\text{eff}} \ekl{\kl{\vec{v}_i^{\,\text{before}}-\vec{v}_j^{\,\text{before}}}\cdot\vec{n}}\vec{n}
\end{equation}
with $m_{eff}=\frac{m_i m_j}{m_i+m_j}$ being the \emph{effective mass}. If $m_i=m_j$ the collision event boils down to the simple relation
\begin{align}
{\vec{v}_i}^{\,\text{after}} &= {\vec{v}_{i}}^{\,\text{before}} - {\vec{v}_{ij}}^{\,n}\cdot\vec{n} \\
{\vec{v}_j}^{\,\text{after}} &= {\vec{v}_{j}}^{\,\text{before}} + {\vec{v}_{ij}}^{\,n}\cdot\vec{n}
\end{align}
The values can be stored once in a look-up table such that there is no need of calculating the correction to the velocities at every collision.








\subsubsection*{Collision with rotation:}

If particles collide with a nonzero tangengial velocity they can exchange angular momentum due to friction. This, together with linear and angular momentum conservation has to be taken into account. As known from classical physics, the equation of motion for rotations holds:


\begin{equation}
I\der{\vec{\omega}_i}{t} = \vec{r}\times\vec{f}_{i,j} = m\vec{r}\times\der{\vec{v}_i}{t},
\end{equation}
with the moment of inertia $I$. If we consider the collision between spheres having the same radius $R$, moment of inertia $I$ and mass $m$ the exchange of angular momentum is given by

\begin{align*}
I\kl{\vec{\omega}_i\,'-\vec{\omega}_i} &= -R m \kl{\vec{v}_i\,'-\vec{v}_i}\times\vec{n}\\
I\kl{\vec{\omega}_j\,'-\vec{\omega}_j} &=  R m \kl{\vec{v}_j\,'-\vec{v}_j}\times\vec{n}
\end{align*}
with the primed velocities being the ones after the collision. Together with the conservation of linear momentum

\begin{equation*}
\vec{v}_i'+\vec{v}_j\,' = \vec{v}_i+\vec{v}_j,
\end{equation*}

one obtains the rule for calculating the new angular velocity after the collision: 

\begin{equation}
\vec{\omega}_i\,'-\vec{\omega}_i = \vec{\omega}_j\,'-\vec{\omega}_j = \frac{R m}{I} \kl{\vec{\vec{v}}_i\,'-\vec{\vec{v}}_i} \times \vec{n}
\label{eq:ang_vel_diff}
\end{equation}

We can decompose the relative velocity $\vec{u}$ of the particles into normal ($\vec{u}^{\,n}$) and tangential ($\vec{u}^{\,t}$) velocity. Mind that we are not interested in the relative velocities of the centers of mass of the particles. What is important for the transfer of angular momentum is the relative velocity of the particles' surfaces at the contact point.
 
\begin{align*}
\vec{u}_{i,j}^{\,n} &= \kl{\vec{u}_{i,j} \vec{n}} \vec{n}  \\
\vec{u}_{i,j}^{\,t} &= \vec{u}_{i,j} \times \vec{n} = \ekl{\kl{\vec{v}_{i}-\vec{v}_{j}} - R\kl{\vec{\omega}_{i}+\vec{\omega}_{j}}}\times\vec{n}
\end{align*}

We can now introduce an artificial parameter that describes the general slip condition:

\begin{equation}
\vec{u}_{i,j}^{\,t}\,' = e_t \vec{u}_{i,j}^{\,t} 
\label{eq:slipc}
\end{equation}


The perfect slip collision is recovered by setting  $e_t=1$, which means that no rotation energy is transferred from one particle to the other, or the extreme case in which there is no slip at all: $e_t=0$. Energy conservation of the system holds only if $e_t=1$, for $ e_t<1$ energy is dissipated. In case of negative coefficient we speak of \emph{superelasticity}.

If we compute the difference of the relative tangential velocities before and after the slip we get

\begin{align*}
\vec{u}_{i,j}^{\,t}-\vec{u}_{i,j}^{\,t}\,' 
&=
\ekl{\kl{\vec{v}_{i}-\vec{v}_{j}} - R\kl{\vec{\omega}_{i}+\vec{\omega}_{j}}}\times\vec{n}
-
\ekl{\kl{\vec{v}_{i}'-\vec{v}_{j}'} - R\kl{\vec{\omega}_{i}'+\vec{\omega}_{j}'}}\times\vec{n}\\
&=
\ekl{\kl{\vec{v}_{i}'-\vec{v}_{i}-\vec{v}_{j}'+\vec{v}_{j}} - R\kl{\vec{\omega}_{i}'-\vec{\omega}_{i} + \vec{\omega}_{j}'-\vec{\omega}_{j}  }}\times\vec{n}\\
\end{align*}
And using \eqref{eq:ang_vel_diff} we get an expression without angular velocities:

\begin{align*}
\vec{u}_{i,j}^{\,t}-\vec{u}_{i,j}^{\,t}\,' 
&=
\kl{1-e_t}\vec{u}_{i,j}^{\,t} \\
&=-\ekl{2\kl{\vec{v}_{i}'-\vec{v}_{i}} + 2q \kl{\vec{v}_{i}'-\vec{v}_{i}} }\times\vec{n}
\end{align*}
\vspace{-0.5cm}
\begin{equation}
\Rightarrow
\hspace{0.2cm} 
\vec{v}_{i}^{\,t}\,' = \vec{v}_{i}^{\,t} - \frac{\kl{1-e_t} \vec{u}_{i,j}^{\,t} }{2\kl{1+q}}, 
\hspace{0.3cm}
\text{with }
q\equiv\frac{mR^2}{I}
\end{equation}
Analogously one finds that
\begin{align}
\vec{v}_{j}^{\,t}\,' &= \vec{v}_{j}^{\,t} + \frac{\kl{1-e_t} \vec{u}_{i,j}^{\,t} }{2\kl{1+q}} \\
\vec{\omega}_{i}^{\,t}\,' &= \vec{\omega}_{i}^{\,t} - \frac{\kl{1-e_t} \vec{u}_{i,j}^{\,t}\times\vec{n} }{2R\kl{1+q^{-1} }} \\
\vec{\omega}_{j}^{\,t}\,' &= \vec{\omega}_{j}^{\,t} + \frac{\kl{1-e_t} \vec{u}_{i,j}^{\,t}\times\vec{n} }{2R\kl{1+q^{-1} }} 
\end{align}

Using conservation of energy, angular and linear momentum we can compute the change in the momentum:
\begin{equation}
\vec{\Delta p} = - 2 m_{\text{eff}} \ekl{ \kl{\vec{v}_{i,j} \vec{n} }\vec{n} + \frac{I}{I+m_{\text{eff}}R^2}  \kl{\vec{v}_{i,j} \vec{s} }\vec{s}  }
\end{equation}

\subsection{Inelastic Collisions}


When particles interact and collide there is always some loss of kinetic enery due to plasticity, deformation, friction, thermal dissipation etc. An example can be a rubber ball that bounces against the floor and does not reach the same height after the collision. Instead of calculating explicitely the loss of kinetic energy due to all the interactions one can introduce an artificial parameter, the \emph{restitution coefficient} to describe the mentioned effects. The coefficient is defined as the ratio of the energy before and after the event, and it can describe multiple physical effects. E.g., the capability of a dice throw to generate random numbers crucially dependson the restitution coefficients \citep{nagler_dice}. In the case of a bouncing ball these include friction with the air, deformation of the ball, heating, etc.:
\begin{equation}
r = \frac{E^{\text{after}}}{E^{\text{before}}} = \kl{\frac{v^{\text{after}}}{v^{\text{before}}}}^2
\end{equation}
As we did before, we can also separate normal and tangential energy transfer, and assign a restitution coefficient to each of them:
\begin{align}
e_n &= \sqrt{r_n} = \frac{v_n^{\text{after}}}{v_n^{\text{before}}}\\
e_t &= \sqrt{r_t} = \frac{v_t^{\text{after}}}{v_t^{\text{before}}}
\end{align}
These coefficients are strongly dependent on the material, the shape of the particles, the energies involved in the events, the angle of impact and other factors. Usually, they are determined experimentally.








As we did before, we can calculate the normal component of the relative velocity the particles at their contact point:
\begin{equation}
\vec{u}_{i,j}^{\,n} =  \kl{\vec{u}_{i,j}\vec{n}} \vec{n} = \ekl{\kl{\vec{v}_{i}-\vec{v}_{j}}\vec{n}}\cdot\vec{n}
\end{equation}
This will be the velocity affected by the inelasticity. In the case of an inelastic collision, dissipation reduces the velocity which after that collision can be expressed by


\begin{equation}
\vec{u}_{i,j}^{\,n}\,' = e_n \vec{u}_{i,j}^{\,n}
\end{equation}
If $e_n$ is equal to 1, then there is no dissipation, if it is smaller we have dissipation effects. The last expression is the same as \eqref{eq:slipc}, and the calculations that follow are also the same. From the conservation of linear and angular momentum, we obtain the expressions for the velocities of each particle after the collision:

\begin{align}
\vec{v}_{i} \,' &= \vec{v}_{i} - \frac{\kl{1+e_n} }{2}\vec{u}_{i,j}^{\,n}\\
\vec{v}_{j} \,' &= \vec{v}_{j} + \frac{\kl{1+e_n} }{2}\vec{u}_{i,j}^{\,n}
\end{align}

In the case of perfect slip we get:

\begin{equation}
\vec{\Delta p}_{n} = -m_{\text{eff}}(1+e_n)\ekl{\kl{\vec{v}_{i}-\vec{v}_{j}}\cdot\vec{n}}\vec{n}
\end{equation}




Putting the results together, for $q\equiv\frac{m_{\text{eff}}R^2}{I_{\text{eff}}}$ we get the equations for the velocities after the collision:


\begin{align}
\vec{v}_{i} \,' &= \vec{v}_{i} - \frac{\kl{1+e_n}  }{2} \vec{u}_{i,j}^{\,n} -  \frac{\kl{1-e_t} \vec{u}_{i,j}^{\,t} }{2\kl{1+q}} \\ 
\vec{v}_{j} \,' &= \vec{v}_{j} + \frac{\kl{1+e_n}  }{2} \vec{u}_{i,j}^{\,n} +  \frac{\kl{1-e_t} \vec{u}_{i,j}^{\,t} }{2\kl{1+q}} \\ 
\vec{\omega}_{i}\,' &= \vec{\omega}_{i} - \frac{\kl{1-e_t} \vec{u}_{i,j}^{\,t}\times\vec{n} }{2R\kl{1+q^{-1} }} \\
\vec{\omega}_{j}\,' &= \vec{\omega}_{j} + \frac{\kl{1-e_t} \vec{u}_{i,j}^{\,t}\times\vec{n} }{2R\kl{1+q^{-1} }} 
\end{align}


Inelastic collisions are very important since in nature situations in which the dynamics of a system can be approximated with perfect elastic mechanics are very rare or practically non-existent. As a simple example for the importance of this subject in computational physics, we will present a phenomenon called \emph{inelastic collapse}. It describes how particle can form cluster if their interaction is not perfectly elastic. In regions of higher particle density, there will be more dissipation of kinetic energy and the particles will tend to slow down in average, thus increasing locally the density even more. Taking this into account, one can simulate the dynamics of galaxies. Without this effect, stars would not be clustered in the way they are in the universe.

\subsection{Inelastic Collapse and TC Model}
\label{sec:collapse}


As shown by McNamara \citep{inelastic_collapse1, inelastic_collapse2}, inelastic collision  can create finite time singularities. This effect can be of great importance in realistic situations like high density gases. If two particles are approaching and there is a third particle bouncing between them, it can be that the system reaches a singularity. 

To understand the effect with a simple model, we will just imagine a ball bouncing vertically on a hard surface. If one tries to compute the motion of an inelastic sphere bouncing on a plate with the model introduced, every time the sphere hits the plate there is the effect of the restitution coefficient to take into account.  Since the kinetic energy is dissipated, after every event the velocity is reduced by the restitution factor, hence the ball cannot reach the same height from which it was dropped. Thus the time between the events approaches zero. After a finite time, the ball comes to a rest, but the simulation would take infinite time to run. This happens because in the event driven model, the ball never stops its motion completely and per $\Delta t$ the event number increases. A similar problem is the famous \emph{Zenon Paradox} \citep{zeno}.


\begin{figure}[h!]
  \centering
  \includegraphics[width=.85\textwidth]{pics/hardpart_comp}
  \captionof{figure}{Comparison of soft particle (left) and hard particle (right) collision.}
  \label{fig:comp_tcmolde}
\end{figure}



\vspace{0.2cm}
If we try to calculate the total time that the ball takes to come at rest, we have to sum up over infinite events each separated by twice the time that the ball takes to hit the ground, $t_j$ (first the ball has to rise and then it falls). Since the height is directly proportional to the energy, also the height scales with the restitution factor at every bounce.  Due to this, at the $j^{th}$  bounce we will have a damping of the height proportional to  $r^j$. With Newtonian mechanics we can compute the time that the ball needs to cover that distance:



\begin{align*}
t_{tot} &= \sum_{j=1}^\infty {t_j}\\
&= 2 \sqrt{\frac{2 h^{\text{initial}}}{g}}  \sum_{j=1}^{\infty} {\sqrt{r^j}}\\
&= 2\sqrt{\frac{2 h^{\text{initial}}}{g}}\kl{\frac{1}{1-\sqrt{r}}-1}
\end{align*}

%10b -5min

The problem in the method lies in the assumption that the interactions are instantaneous. On the contrary, real collisions have a certain duration. Luding and McNamara \citep{ludingmcnamara} introduced a coefficient of restitution that is dependent of the time elapsed since the last event occurred. If the last collision for one of the event partners is less than $t_{\text{contact}}$, then the coefficient is set to one:
%(BLABLA  both??? fehler in folie 44, da heisst es "both particles") 
\begin{equation}
r^{i,j} =\begin{cases}
  r,  & \text{for } t^{i}>t_{\text{contact}} \text{ or } t^{j}>t_{\text{contact}}\\
  1, & \text{otherwise }.
\end{cases}
\end{equation}

Depending on the materials that one wants to simulate more complex dependencies can be used. If , for example, one has a very dense and viscous material 0 instead of 1 can be used. In this case the particles form clusters and stick together.

Comparing soft potential in molecular dynamics and the event driven hard particles (see Fig. \ref{fig:comp_tcmolde}) we see the main difference. In the first case the collision was discretized, while in event driven we have strict binary collision that are instantaneous. In molecular dynamics the condition is softened and multiple collision may occurr in the same time.
          





















%------------------------------------------------------------------------------------------------------------------------------------------
%\begin{comment}
\include{chapters/inelastic_collision/inelastic_collision}
\include{chapters/inelastic_collision/contact_dynamics}
\include{chapters/inelastic_collision/particle_in_fluids}
%------------------------------------------------------------------------------------------------------------------------------------------
\include{chapters/canonical_ensemble/canonical_ensemble}
%------------------------------------------------------------------------------------------------------------------------------------------
\chapter{Quantum Mechanics}


Apart from particle physics, one is generally not interested in single particles but in a system composed by many particles.  Since simulations in particle physics require the knowledge of some theories which are not necessarily known by the reader (e.g., QFT), we will here reduce the discussion to ``classical'' quantum mechanics. In quantum mechanics the laws of physics differ very much from the laws we usually make use of in the macroscopic world. Here we will introduce some concepts used for the simulation of such regimes. This is not a lecture on QM, but we will briefly mention the main concepts used in the simulation of quantum mechanical systems. \textbf{Wave functions} for particles are defined, as well as the time independent \textbf{ Schr\"odinger equation}.  Basic expansions and approximations (in particular, \textbf{Born-Oppenheimer} and \textbf{Kohn Sham}) follow the definitions and represent the core of the chapter. A quick overview of the \textbf{Car-Parrinello} method concludes this little excursus in the world of quantum mechanics.

If the reader is not familiar with QM, dedicated lectures are necessary to fully understand what is done in this chapter. Should the reader be interested in some deeper insights in the simulation of quantum mechanical systems, there are lectures such as \emph{Computational Quantum Physics} that are also recommended as an integration of this course.

\section{Introduction}

Particles are described by complex \emph{wave functions}, usually denoted with $\psi$, which depend on time and space $t,\vec{r}$. The probability of finding a particle at some point in space and time is given by the absolute squared value of $\psi$:
\begin{equation}
\rho\kl{\vec{r},t} = \abs{\psi\kl{\vec{r},t}}^2 = \psi^*\kl{\vec{r},t}\psi\kl{\vec{r},t}.
\end{equation}
$\rho$ is called the \emph{probability density} and is a normalized function such that
\begin{equation}
\int \rho\kl{\vec{r},t} \text{d}\vec{r}\,\text{d}t = 1,
\end{equation}
i.e., the probability of finding a particle somewhere at some point in time is equal to 1. This does not generally mean that particles or composed objects (like the reader of this script) are likely to be found everywhere at random distances in space or time. Even though we very often use plane waves for a qualitative description of quantum mechanical effects, ``real'' wave functions can be localized more or less sharply (we speak of \emph{wave packages}). This means that wave functions (and in particular probability densities) converge fastly towards zero when we leave the coordinates of interest.


These wave functions $\psi$ are eigenfunctions of the quantum mechanical \emph{Hamilton operator} $\mathcal{H}$. They are a solution of the \emph{Schr\"odinger equation}, that describes the energy and the time evolution of the particles. For example, the time independent Schr\"odinger equation is given by
\begin{equation}
\mathcal{H}\psi\kl{\vec{r}} = E \psi\kl{\vec{r}}.
\end{equation}
where E is the energy of a state and the Hamiltonian represents the kinetic and potential energy of the particle in this particular state:
\begin{equation}
\mathcal{H} = - \frac{\hbar^2}{2m} \nabla^2  +V\kl{\vec{r}}.
\end{equation}
From the Hamiltonian (in the $\frac{\hbar}{m}$ factor) one can see that quantum mechanical effects disappear for \emph{large} masses. Already for the mass of ions quantum mechanical effects can be neglected and they can be approximatively treated with classical physics.

The ground state is defined as the state with the lowest energy eigenvalue. Here we will only be concearned about ground states and equilibrium states. For a composed object, the wave function depends on all the composing particles:

\begin{equation}
\rho\kl{\vec{r_1},\vec{r_2},...,\vec{r_N},t} = \abs{\psi\kl{\vec{r_1},\vec{r_2},...,\vec{r_N},t}}^2 = \psi^*\kl{\vec{r_1},\vec{r_2},...,\vec{r_N},t}\psi\kl{\vec{r_1},\vec{r_2},...,\vec{r_N},t}
\end{equation}

\noindent
with $\rho$ again being the probability of finding the particles at a certain point in time, normalized over all coordinates and time. 

If the wave function is symmetric under exchange of two particles (+), we call the object a \emph{Boson}, if the function is antisymmetric (-), a \emph{Fermion}:


\begin{equation}
 \psi\kl{\vec{r_1},...,\vec{r_i},...,\vec{r_j},...,\vec{r_N},t} = \pm \psi\kl{\vec{r_1},...,\vec{r_j},...,\vec{r_i},...,\vec{r_N},t}.
\end{equation}

Already from this one can see that fermions behave very differently from classical particles: exchanging composing particles leads to a wave function with opposite sign! One of the consequences of this asymmetry is that identical fermions (e.g. two electrons) will never occupy the exact same state at the same coordinates in space and time (\emph{Pauli exclusion principle}).

If the particles composing the objects are not interacting or are only loosely correlated, one can approximate the wave function using the \emph{Slater determinant} and the wave functions of the free particles:

\begin{equation}
\psi\kl{\vec{r_1},...,\vec{r_N}}
=
\frac{1}{N!}
\abs{
\begin{pmatrix}
 \psi_1\kl{\vec{r_1}} & \cdots & \psi_N\kl{\vec{r_N}}\\
 \vdots				  & \ddots & \vdots	\\
 \psi_1\kl{\vec{r_1}} & \cdots & \psi_N\kl{\vec{r_N}}\\
\end{pmatrix}
}
\label{eq:slater}
\end{equation}


Eventual corrections for the correlation between the components of the object can be added to the equations. The calculation of \eqref{eq:slater} is computationally very expensive and time consuming.









\section{Approximations in Quantum Mechanics}


\subsection{Implementation Of Wave Functions}

An interesting approach to simulate quantum systems is the the direct implementation of wave functions. This is a conditional method, since often the system is so complex that it is very difficult to simulate the entire set of degrees of freedom. Complex molecules are already very difficult to fully simulate with this method. The most used and simplest approach is to expand the wave function in an orthonormal basis system and cut the expansion at some point. Plane waves are often used for this:



\begin{equation}
\psi_j = \sum_k c_{jk} \xi_k
\end{equation}
with $\xi_k = \exp\kl{ikx}$. Typically one looks at how many plane waves are needed until the energy converges (a typical number would be around some thousands plave waves). Another option is to take localized waves instead of plane waves. Paul Pulay did this in 1969 using Gaussian orbitals:

\begin{equation}
\xi_l\kl{r} = c_l r^l \exp\kl{-\alpha r^2}
\end{equation}

Due to the cut-off of the basis of the  wave function, one gets the so called \emph{Pulay forces} which are a numerical artefact that has to be corrected. Its effect can reach the order of magnitude of the real physical forces and is therefore not neglegible.








\subsection{Born-Oppenheimer Approximation}

A more common approach for atoms and molecules is given by the \emph{Born-Oppenheimer} approximation: the core of the atoms and the electrons can be decoupled, since the masses are very different ($\propto$1000 times). Because of the $\frac{\hbar}{m}$ factor, the atomic core can be first treated classically with MD and after having solved their motion the electrons can be treated quantum mechanically.

The limitation  of this approach is that the electrons can change their energy state (e.g., in atomic transitions), and the condition for the Born Oppenheimer approximation is thus that the motion of the ions is discretized in time steps small enough such that these transitions are resolved.

\subsubsection*{Density function theorem}
If the ground state is not degenerated, the approximation that physical quantities only depend on the probability density and not on the wave function can be made. It is thus enough to consider the square of the wave function to calculate the quantities and the wave function itself is not needed.


\subsection{Kohn-Sham Approximation}
If one develops further the idea that particles can be treated separately, non interacting single particles moving in a potential can be considered, e.g. when treating electrons. After the density function theorem, our quantities only depend on the density distribution. After having solved the motion or the quantities of interest following the mentioned approximation, correlation effects can be corrected with further terms. 

Taking the orthonormality condition of the basis wave functions into account 

\begin{equation}
\int{  \psi_i^*\kl{\vec{r}} \psi_j\kl{\vec{r}}  \text{d}r } = \delta_{ij}, 
\end{equation}


\noindent
one can formulate the problem separating the potential in which the electrons move from their kinetic and interaction part:


\begin{equation}
E_{KS} = 
\underbrace{- \sum_i {  f_i \int{\psi_i^*\kl{\vec{r}} \frac{\hbar^2}{2m_i}\nabla^2\, \psi_i\kl{\vec{r}} \text{d}\vec{r}  }    } }_{\text{kinetic energy}}  + \underbrace{ V_{\text{eff.}} }_{\text{pot. energy}}.
\end{equation}


\noindent
With the occupation number $f_i$ and the effective potential


\begin{equation}
V_{\text{eff}}\kl{\rho} =
\underbrace{\int{\rho\kl{\vec{r}} V_{\text{ext.}}\kl{\vec{r}}  \text{d}\vec{r} }}_{\text{ions term}} +
\underbrace{\frac{1}{2} \int\int \frac{ \rho\kl{\vec{r}} \rho\kl{\vec{r}\,'} }{\abs{\vec{r}-\vec{r}\,'}} }_{\text{Coulomb term}}
\text{d}\vec{r}\text{d}\vec{r}\,' +
\underbrace{E_{\text{exc.}} + E_{\text{corr.}}}_{\text{correction terms}}
\end{equation}


\noindent
given by the external core field, the Coulomb term of the electronic interaction and the correction terms for quantum mechanical effects of the electrons (change of state, correlation, fermion statistics, etc.). Note how everything only depends on the probability density

\begin{equation}
\rho\kl{\vec{r}} = 
\sum_i {  f_i \int{  \psi_i^*\kl{\vec{r}} \psi_i\kl{\vec{r}}  \text{d}r }  }.
\end{equation}



Due to the effect that everything has been boiled down to single particles functions in an external potential, this particle functions are not physical quantities  anymore, but terms in a decomposition. The sum of the energies of the single particles is therefore not equal the total energy of the system!

The exchange energy is still to be calculated. Using the Slater determinants one can calculate the free electron gas analutically. In the \emph{Local Density Approximation} (LDA) the correction terms in the Kohn-Sham approximation are given by a free electron gas with the same density\footnote{This result, like all the other concepts treated in this chapter, is derived in quantum mechanics lectures and every introductory literature on quantum mechanics.}:

\begin{equation}
E_x^{\text{LDA}} = 
-\frac{3}{4} \kl{\frac{3}{\pi}}^{\frac{1}{3}} \int {\rho\kl{\vec{r}}^{\frac{4}{3}} \text{d}\vec{r}  }
\end{equation}

\noindent
This represents the energy cost of having some electrons in a certain coordinate. This is only an approximation and has also to be corrected further.

Since single body wave functions are used, the neglected correlation effects between them also have to be corrected. This happens in $E_c$. Again, using LDA, one can assume that

\begin{equation}
E_c^{\text{LDA}} = 
 \int {\epsilon_c \kl{\rho\kl{\vec{r}}} \text{d}\vec{r}  }
\end{equation}
Where $\epsilon_c$ is a term for which no analytic expression is known.


LDA is usually applied to calculations in band structures and solid state physics, where the approximation of a free electron gas is accurate. In quantum chemistry, where the chemical bonds are more sensitive to the density, this is not enough. To do a better approximation one also has to take  the gradient of the probability density into account. LDA is thus improved by adding a dependence on the gradient of $\rho$. This is called the \emph{General Gradient Approximation} (GGA).  There is a good physical reason for this correction: quantum mechanical effects are very strong when there is a change in the slope of the wave function and in particular when two identical fermions come closer. Due to the Pauli exclusion principle, non classical effects are stronger if the particles are close. The correction term is thus given by:
\begin{equation}
E_x^{\text{GGA}} = 
 \int {\epsilon_x \kl{   \rho\kl{\vec{r}}, \abs{\nabla  \rho\kl{\vec{r}} }   } \text{d}\vec{r}  }
\end{equation}
Where $\epsilon_x$ is a term for which no analytic expression is known. There are several attempts to give a functional form for the GGA, e.g. the one proposed by Frank Herman in 1969:

\begin{equation}
E_x^{\text{GGA}} =  E_x^{\text{LDA}} -
 \beta \int { \frac{\kl{\nabla \rho}^2}{\rho^{\frac{4}{3}}}  \text{d}\vec{r}  }
\end{equation}
with $\beta$ being a numerical constant or the more used one, by a professor in Halifax (Canada), Axel Becke \citep{becke}:
 
\begin{equation}
E_x^{\text{Becke}} =  E_x^{\text{LDA}} -
 \beta \int {\rho^{\frac{4}{3}} \frac{x^2}{1+6\beta x \text{sinh}^{-1}\kl{x}}  \text{d}\vec{r}  }
\end{equation}
with $x\equiv \frac{\abs{\nabla\rho}}{\rho^{\frac{4}{3}}}$ and $\beta = 0.0042$. The fact that his paper with this correction is the most cited paper in physical science (to date) gives an idea of the importance of this subject. This is not an analytically derived formula, it is obtained purely empirically. It fails on certain regimes (e.g. it doens't fully recover the Van der Waals forces correctly). This has been corrected further, e.g. by Grimme \citep{grimme}.




\subsection{Hellmann-Feynman Theorem}


Once the electrons are solved, one can iterate and solve the motion of the ions in the ion-electron potential. Supposing we have the Hamiltonian of the system, then the forces acting on the ions are given by the expectation value of the derivative of the Hamiltonian with respect to the spatial coordinates (\emph{Hellmann-Feynman Theorem}). This theorem, which will not be proven here, can be written as \footnote{Here we use the \emph{Bra-Ket} notation: $\avkl{\psi\abs{A}\psi} \equiv \int{\psi^* A \, \psi}$. For further information consult literature on quantum mechanics.}
\begin{equation}
m_{\alpha} \dder{\vec{R}_{\alpha}}{t} =
- \avkl{\psi\abs{\pder{\mathcal{H}}{\vec{R}_{\alpha}}}\psi}.
\label{eq:hell-fey}
\end{equation}

What we have constructed here is an iterative method for solving the motion of molecules or similar objects by separating the wave functions. Summarizing, the receipe is the following:
\begin{itemize}
\item Solve the electronic motion in the effective potential of the ions by separating the wave functions.
\item Solve the motion of the ions using classical MD simulations together with the forces given by the Hellmann-Feynman theorem \eqref{eq:hell-fey}.
\item Iterate
\end{itemize}

This seems a reasonable method, but the problem is that the first step can be very difficult. Molecules or composed particles can assume very complex form and the degrees of freedom for the electronic cloud are usually tens or hundreds, making the computation very cumbersome. For this, the contribution of Car and Parrinello has been very important (see next section).




\section{Car-Parrinello Method}


Car and Parrinello managed to cast the eigenvalue problem into a molecular dynamics problem for the electrons \citep{carpar1}. They constructed a Hamiltonian for the entire system:

\begin{equation}
\mathcal{H}_{CP} \kl{R^n, \dot{R}^n, \mkl{\psi}, \mkl{\dot{\psi}} } 
=
\sum_l{   \underbrace{ \frac{1}{2}\frac{P_l^2}{M_l}}_{\text{kin. en. ions}}   }
+ \,
\sum_l{  \underbrace{  \frac{\mu}{2}\avkl{\dot{\psi}|\dot{\psi}} }_{e^-\text{ kin. en.}}   }
+\,
\underbrace{E^{\text{KS}}\kl{R^n,\mkl{\psi}}}_{\text{Kohn-Sham energy}}
\,+
\underbrace{E^{\text{Ion}}\kl{R^n}}_{\text{ion-ion inter. en.}}
\end{equation}


Similar to the Nos\'e-Hoover approach, we regard the ions and the electrons as two different system coupled with a Hamiltonian, that consists of the kinetic and potential parts of both ions and electrons. The coupling however, is done with an artificial mass, $\mu$ which is not the real mass of the electrons. The kinetic energy of the electronic part is in reality not equal the kinetic energy of the physical electrons\footnote{Mind that even the electrons wave functions $\psi$ are not the functions of the real electrons!}. This artificial electronic kinetic energy, given by $\mu$, is a numerical parameter which represents the coupling of the two systems\footnote{The analogy with the Nos\'e-Hoover thermostat might now be clearer: recall the coupling variable $Q$ that described how strong the heath bath and the particles were coupled.}. The higher $\mu$ is, the more the ions and the electrons are coupled and the faster is the electronic response to the ionic movement. The motion of the electrons can now be found using the hamiltonian and integrating numerically. Mind that the integration time step has to be small enough to resolve the electronic motion. 


The main advantages of the Car-Parrinello method are that the computation is much faster and the energy fluctuations are smaller compared to the classical Born-Oppenheimer approximation. However, for small atoms and molecules (e.g. the hydrogen atom), the light ions and the relatively high $\mu$ lead to numerical problems and deviations from the real physical quantities.

\vspace{0.7cm}


All these methods described here rely on approximations and cannot be seen as a realistic description of the physical processes happening on quantum mechanical scales. As the exotic methods described in fluid dynamics, they have to be taken as numerical methods that give accurate results. Even if they are inspired by reality, they do not necessarily reflect it. The fact that one can implement wave functions using plave waves basis function, as well as gaussian basis function both with a cut-off is emblematic. Depending on the situation one might use one or the other. For small and isolated molecules (i.e., spatially localized) one needs many basis functions and has to pay careful attention to the cut-off due to energy-related problems (e.g., the Pulay forces for the Gaussian basis wave functions). On the other way, plane waves are analytically easier and simpler to implement with existing scientific libraries. The decision of which method has to be used does not always imply that one is more realistic than the other, only that it gives better and/or faster results.

















%\end{comment}
%%%%%%%%%%%%%%%%%%%%%%%%%%%%%%%%%%%%%%%%%%%%%%%
%%%%%%%%%%%%%%%%%%%%%%%%%%%%%%%%%%%%%%%%%%%%%%%
%%%%%%%%%%%%%%%%%%%%%%%%%%%%%%%%%%%%%%%%%%%%%%%






%% literaturverzeichnis
\bibliographystyle{plainnat}
\bibliography{chapters/shit/literatur}

\appendix


\include{chapters/shit/appendix}
\include{chapters/shit/danksagung}



\end{document}






% $Id: makros.tex 1131 2008-09-29 20:25:00Z doerr $


% Macro zum kommentieren

\newcommand{\comm}[1]{\begin{comment} #1 \end{comment}}


\newlength{\extlw }
\setlength{\extlw}{\linewidth}   % Saving fboxsep into mylength
\newlength{\exttw }
\setlength{\exttw}{\textwidth}   % Saving fboxsep into mylength


% Macros zum vereinfachen von partiellen und vollständigen
% Ableitungen
\newcommand{\pder}[2]{\ensuremath\frac{\partial #1}{\partial #2}}
\newcommand{\der}[2]{\ensuremath{\frac{\text{d} #1}{\text{d} #2}}}

\newcommand{\pdder}[2]{\ensuremath\frac{\partial^2 #1}{\partial #2 ^2}}
\newcommand{\dder}[2]{\ensuremath{\frac{\text{d}^2 #1}{\text{d} #2 ^2}}}

\newcommand{\fixme}[1]{\textcolor{red}{(#1)}%
  \marginpar{\textcolor{red}{Fix Me!}}%
%  \pdfmarginnote[bla]{#1}
}

\newcommand{\todo}[1]{\textcolor{blue}{(#1)}%
  \marginpar{\textcolor{blue}{Todo}}%
%  \pdfmarginnote[bla]{#1}
}

        

% neues Macro für h-quer. Ließt sich besser als hbar
\newcommand{\hquer}{\ensuremath{\hbar}}

% matritzen und vectoren sind fett, matritzen mit hut
%\newcommand{\mat}[1]{\ensuremath{\boldsymbol{\mathcal{#1}}}}
\newcommand{\mat}[1]{\ensuremath{\overset{\leftrightarrow}{#1}}  }   
%\renewcommand{\vec}[1]{\ensuremath{\boldsymbol{\mathrm{#1}}}}


% Differential in Integralen (das 'd' in dt soll nicht
% kursiv sein
\newcommand{\dif}[1]{\ensuremath{\text{d}#1}}
\newcommand{\lint}[4]{\ensuremath{\int\limits_{#1}^{#2}#3\text{d}#4}}
\newcommand{\lintb}[2]{\ensuremath{\int_{#1}^{#2}}}

% erwartungswert
\newcommand{\expect}[1]{\ensuremath{\langle #1 \rangle}}

% mathematische symbole
\newcommand{\rot}{\ensuremath{\text{rot}}}
\newcommand{\divg}{\ensuremath{\text{div}}}
\newcommand{\FSR}{\ensuremath{\text{FSR}}}
\newcommand{\FWHM}{\ensuremath{\text{FWHM}}}
\newcommand{\kreuz}{\ensuremath{\times}}
\newcommand{\skalar}{\ensuremath{\cdot}}

\newcommand{\abs}[1]{\left|#1\right|}
\newcommand{\avkl}[1]{\left<#1\right>}
\newcommand{\mkl}[1]{\left\{#1\right\}}
\newcommand{\kl}[1]{\ensuremath{\left(#1\right)}}
\newcommand{\ekl}[1]{\ensuremath{\left[#1\right]}}

\newcommand{\halb}{\ensuremath{\frac{1}{2}}}
\newcommand{\gwaved}{\emph{gwaved}}
\newcommand{\ren}{\emph{renac}}


% 2x1 spaltenvektor mit klammern
\newcommand{\cvec}[2]{\ensuremath{%
        \begin{bmatrix}
        #1\\
        #2
        \end{bmatrix}}}

% 3x1 spaltenvektor mit klammern
\newcommand{\cvvec}[3]{\ensuremath{%
        \begin{bmatrix}
        #1\\
        #2\\
		#3
        \end{bmatrix}}}


% 4x1 spaltenvektor mit klammern
\newcommand{\cvvvec}[4]{\ensuremath{%
        \begin{bmatrix}
          #1\\
          #2\\
          #3\\
          #4
        \end{bmatrix}}}

% 4x1 spaltenvektor mit rundenklammern
\newcommand{\cvvecr}[4]{\ensuremath{%
        \begin{pmatrix}
          #1\\
          #2\\
          #3\\
          #4
        \end{pmatrix}}}

% 2x2 matrix mit klammern
\newcommand{\pmat}[4]{\ensuremath{%
        \begin{bmatrix}
          #1 &  #2\\
          #3 &  #4
        \end{bmatrix}}}


% osterei

\newcommand{\easteregg}[1]{%
        \ifdefined\printEasterEggs
        #1
        \fi}

\newcommand{\JonesM}{\ensuremath{\mat{J}}}


\renewcommand{\Re}{\ensuremath{\text{Re}}}

%%%%%%%%%%%%%%%%%%%%%%%%%%%%%%%%%%%%%%%%%%%%%%%%%%%%%%%%5
%%% Abkürzungen

\newcommand{\fabryperot}{Fa\-bry-P\'{e}rot}
\newcommand{\fpi}{\fabryperot -Inter\-fer\-ome\-ter}

\newcommand{\gw}{gra\-vi\-ta\-tio\-nal wave }
\newcommand{\ex}{\ensuremath{\vec{e}_x} }
\newcommand{\ey}{\ensuremath{\vec{e}_y} }
\newcommand{\kep}{\emph{Kepler} }
\newcommand{\eva}{\emph{eva2 } }
\newcommand{\sun}{\emph{sunday } }
\newcommand{\cm}{\ding{51}}%
\newcommand{\xm}{\ding{55}}%
\makeatletter
\newcommand{\thickhline}{%
    \noalign {\ifnum 0=`}\fi \hrule height 1pt
    \futurelet \reserved@a \@xhline
}
\newcolumntype{"}{@{\hskip\tabcolsep\vrule width 1pt\hskip\tabcolsep}}
\makeatother
\newcommand{\hsp}{\hspace*{0.1cm}}%



%%% helpers
% \newcommand{\fixme}[1]{\textcolor{red}{(#1)}%
%         \marginpar{\textcolor{red}{Fix Me!}}}

        

%%% Local Variables: 
%%% mode: plain-tex
%%% TeX-master: "dipse"
%%% End: 
